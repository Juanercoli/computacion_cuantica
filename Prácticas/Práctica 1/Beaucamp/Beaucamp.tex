\documentclass{scrartcl}
\usepackage[a4paper,margin=2cm,footskip=1cm]{geometry}
\setkomafont{disposition}{\normalfont\bfseries}

%\documentclass{article}

\usepackage[table,xcdraw]{xcolor}
\usepackage{tikz}
\usetikzlibrary{angles,quotes}
\usetikzlibrary{babel}

\usepackage{booktabs}
\usepackage[utf8]{inputenc}
\usepackage[spanish, es-nodecimaldot]{babel}
\usepackage[per-mode=symbol]{siunitx}
\usepackage{graphicx}
\usepackage{subcaption}
\usepackage{caption}
\usepackage{mathtools}
\usepackage{amsmath}
\usepackage{slashed}
\usepackage{dsfont}
\usepackage{float}
\usepackage{multicol}
\usepackage{wrapfig}
\usepackage{lipsum}
\usepackage{textcomp}
\usepackage{gensymb}
\usepackage{longtable}
\usepackage{supertabular}
\usepackage{hhline}
\usepackage{enumerate}
\usepackage{multirow}
\usepackage{amssymb}
\usepackage{tabularx}
\usepackage{ragged2e}
\usepackage{rotating}
\usepackage{cancel}
\usepackage{physics}
\usepackage[framemethod=default]{mdframed}
\usepackage{csquotes}
%\usepackage[backend=biber, style=numeric, sorting=none]{biblatex}


\renewcommand{\figurename}{Figura}
\renewcommand{\spanishtablename}{Tabla}
\newcommand{\inv}[1]{\frac{1}{#1}}
\newcommand{\uv}[1]{\hat{\mathbf{#1}}}
\newcommand{\uvs}[1]{\, \uv{#1}}

\newcommand{\realSet}{\mathbb{R}}
\newcommand{\complexSet}{\mathbb{C}}
\newcommand{\oref}{$\mathcal{O}$ }
\newcommand{\opref}{$\mathcal{O}'$ }
\newcommand{\oppref}{$\mathcal{O}''$ }

\def\residue{\mathop{\text{Res}}}

\setlength{\tabcolsep}{19pt}

\DeclareSIUnit\clight{\text{\ensuremath{c}}}
\DeclareSIUnit\MeV{\mega\electronvolt}
\DeclareSIUnit\GeV{\giga\electronvolt}
\DeclareSIUnit\MeVpc{\MeV\per\clight\squared}
\DeclareSIUnit\GeVpc{\GeV\per\clight\squared}

%\newcommand{\vbeta}{\vb{\beta}}
\newcommand{\sinc}{\text{sinc}}
\newcommand{\E}{\vb{E}}
\newcommand{\B}{\vb{B}}
\newcommand{\x}{\vb{x}}
\newcommand{\y}{\vb{y}}
\newcommand{\z}{\vb{z}}
\newcommand{\p}{\vb{p}}
\renewcommand{\k}{\vb{k}}
\newcommand{\Lag}{\mathcal{L}}
\newcommand{\Ham}{\mathcal{H}}

\newcommand{\tx}{\tilde{x}}

\renewcommand{\a}[1]{\hat{a}_{#1}}
\newcommand{\ad}[1]{\hat{a}_{#1}^\dagger}

\DeclareRobustCommand{\[}{\begin{equation}}
\DeclareRobustCommand{\]}{\end{equation}}
\mathtoolsset{showonlyrefs}

\allowdisplaybreaks

%\bibliography{bibliography}

%----------------------------------------------------------------------------------------
%	DOCUMENT INFORMATION
%----------------------------------------------------------------------------------------

\title{Teoría de la Información Cuántica}
\subtitle{Práctica 1 - Año 2020}
\author{\textsc{Beaucamp}, Jean Yves}
\date{}

\begin{document}

\maketitle

\section{Operador Densidad}
\begin{enumerate}
    
    %-------------------------------------------------------------------------------------------------------
    %   Problema I.1
    %-------------------------------------------------------------------------------------------------------
    \item Sea $\rho$ un operador densidad. Queremos demostrar que
    \[ \rho = \dyad{\psi}{\psi} \iff \rho^2 = \rho. \]
    \begin{itemize}
        \item[$\Longrightarrow$)] Sea $\rho = \dyad{\psi}{\psi}$ (estado puro). Entonces
        \[ \rho^2 = \ket{\psi} \bra{\psi}\ket{\psi} \bra{\psi} = \dyad{\psi}{\psi} = \rho. \]
        \item[$\Longleftarrow$)] Sea un operador densidad $\rho$ que cumpla $\rho^2 = \rho$, con $\rho = \sum_{k} s_k \dyad{k}{k}$ en la base de Schmidt. Luego,
        \[ \rho^2 = \sum_{k, k'} s_k s_{k'} \ket{k} \bra{k} \ket{k'} \bra{k'} = \sum_{k} (s_k)^2 \dyad{k}{k} \stackrel{\rho^2 = \rho}{=} \sum_k s_k \dyad{k}{k}. \]
        Luego,
        \[ (s_k)^2 = s_k \iff s_k(s_k - 1) = 0 \implies s_k = 0, 1. \]
        Pero por la condición de normalización $\Tr{\rho} = 1$, entonces
        \[ s_{\tilde{k}} = 1, \quad s_k = 0 \ \forall k\neq\tilde{k}, \]
        \[ \therefore \rho = \dyad{\psi}{\psi}. \]
    \end{itemize}
    
    
    
    %-------------------------------------------------------------------------------------------------------
    %   Problema I.2
    %-------------------------------------------------------------------------------------------------------
    \item Sean $\rho_i$, $i = 1, \dots, m$ operadores densidad, y definiremos
    \[ \rho = \sum_{i = 1}^m p_i \rho_i, \]
    donde $\p_i \geq 0$, cumpliendo la condición de normalización $\sum_{i = 1}^{m} p_i = 1$. Para ver que $\rho$ es también un operador densidad, verificaremos que cumple las tres condiciones necesarias:
    \begin{itemize}
        \item $\rho$ es un operador hermítico, ya que
        \[ \rho^\dagger = \sum_{i = 1}^m p_i \rho_i^\dagger = \sum_{i = 1}^m p_i \rho_i^. \]
        
        \item $\rho$ se encuentra apropiadamente normalizado:
        \[ \Tr{\rho} = \sum_{i = 1}^m p_i \Tr{\rho_i} = \sum_{i = 1}^m p_i = 1. \]
        
        \item $\rho$ es un operador positivo:
        \[ \mel{\psi}{\rho}{\psi} = \sum_{i = 1}^m p_i \sum_{k = 1}^{n_i} \ip{\psi}{\phi_k^{(i)}} \ip{\phi_k^{(i)}}{\psi} = \sum_{i = 1}^m p_i \sum_{k = 1}^{n_i} \abs{\ip{\psi}{\phi_k^{(i)}}}^2 \geq 0.  \]
    \end{itemize}
    
    
    
    %-------------------------------------------------------------------------------------------------------
    %   Problema I.3
    %-------------------------------------------------------------------------------------------------------
    \item Sea $\rho = p\dyad{0}{0} + (1 - p) \dyad{1}{1}$, con $p \in (1/2, 1)$. Buscamos un cambio de base tal que el operador densidad se encuentre definido como
    \[ \rho' = q\dyad{\alpha}{\alpha} + (1 - q) \dyad{\beta}{\beta}, \]
    con $q \in [1-p, p]$ y $\ket{\alpha}$, $\ket{\beta}$ estados normalizados.
    
    Queremos que en la nueva base $B = \{ \ket{\alpha}, \ket{\beta} \}$ los autovalores del operador densidad diagonal $\rho'$ resultes ordenados $q \leq 1 - q$. Es trivial ver que, para los valores permitidos de $p$, $1 - p \leq p$. Luego, por la definición de la relación de Majorización $\vb{q} \prec \vb{p}$, siendo $\vb{p} = (p, 1 - p)$ y $\vb{q} = (1 - q, q)$ vectores de autovalores de $\rho$ y $\rho'$ en orden descendiente,
    \[ p \geq 1 - q \iff q \geq 1 - p. \]
    Además,
    \[ 1 - q \geq q \geq 1 - p \implies 1 - q \geq 1 - p \implies p \geq q. \]
    Por lo tanto, $q \in [1 - p, q]$. Además, los vectores $\vb{p}$ y $\vb{q}$ estarán relacionados por una matriz doble estocástica $D$:
    \[
        \rho' = D \rho \implies
        \begin{pmatrix}
            1 - q & 0 \\
            0 & q
        \end{pmatrix}
        =
        \begin{pmatrix}
        t & 1 - t \\
        1 - t & t
        \end{pmatrix}
        \begin{pmatrix}
            p & 0 \\
            0 & 1 - p
        \end{pmatrix}.
    \]
    
    
    %-------------------------------------------------------------------------------------------------------
    %   Problema I.4
    %-------------------------------------------------------------------------------------------------------
    \item Queremos expresar la matriz densidad de un sistema de dos qubits como
    \[ \rho = \inv{2} (\mathds{1} + \vb{r} \vdot \vb{\sigma}), \]
    es decir, como una combinación lineal de las matrices $\{ \sigma_0, \sigma_1, \sigma_2, \sigma_3 \}$, siendo $\sigma_0 = \mathds{1}$ y $\sigma_k$ las matrices de Pauli. Sabemos que el conjunto ${\sigma_\mu}$ de las matrices de Pauli más la identidad son una base completa de las matrices de $2x2$, al ser linealmente independientes. Luego, como el operador densidad de un estado general de 1 qubit podrá ser identificado con una matriz de $2x2$ dado por
    \[
        \rho = \inv{\mathcal{N}} \left( \alpha \dyad{0}{0} + \beta \dyad{1}{1} + \xi \dyad{0}{1} + \zeta \dyad{1}{0} \right) \equiv \inv{\mathcal{N}}
        \begin{pmatrix}
            \alpha & \xi \\
            \zeta & \beta
        \end{pmatrix},
    \]
    con $\mathcal{N}$ una constante de normalización tal que siempre se cumpla $\Tr \rho = 1$. Luego, podremos escribir de forma generalizada
    \[ \rho = R_\mu \sigma_\mu = \inv{2} r_\mu \sigma_\mu. \]
    La condición de normalización eliminará uno de los grados de libertad del sistema, por lo que solo tendremos 3 parámetros libres que definan al estado. Eligiendo $\vb{r} = (r_1, r_2, r_3)$ con $\abs{\vb{r}} \leq 1$, entonces utilizamos $r_0$ como la constante de normalización para obtener la traza unitaria:
    \[ \Tr\rho = \frac{1}{2} \left( r_0 \Tr\mathds{1} + r_1 \cancel{\Tr\sigma_1} + r_2 \cancel{\Tr\sigma_2} + r_3 \cancel{\Tr\sigma_3} \right) = \frac{r_0}{2} 2 = \underbrace{r_0 = 1}_{\text{Norm}}. \]
    Luego, 
    \[ \rho = \inv{2} (\mathds{1} + \vb{r} \vdot \vb{\sigma}). \]
    
    Para evaluar los autovalores de $\rho$, podemos rotar el sistema de referencia tal que $\vb{r} \to \vb{r}' = r \vu{z}'$. Luego,
    \[
        \rho' = \inv{2} (\mathds{1} + \vb{r} \vdot \vb{\sigma}) = \inv{2} (\mathds{1} + r \sigma_z') =
        \begin{pmatrix}
            \frac{1 + r}{2} & 0 \\
            0 & \frac{1 - r}{2} \\
        \end{pmatrix},
    \]
    por lo que $\lambda_\pm = (1 \pm r)/2$, y
    \[ \rho = \lambda_+ \dyad{0'}{0'} + \lambda_- \dyad{1'}{1'} \]
    (base de Schmidth). Luego, para $r = 1$ el sistema corresponderá a un estado puro al anularse uno de los autovalores, siendo entonces
    \[ \rho = \lambda_+ \dyad{0'}{0'} + \cancel{\lambda_-} \dyad{1'}{1'} = \dyad{0'}{0'}. \]
    
    Finalmente, como tenemos las identidades de trazas $\Tr(\sigma_\mu \sigma_nu) = 2 \delta_{\mu\nu}$, entonces
    \[ \expval{\sigma_k} = \Tr(\rho \sigma_k) = \inv{2} \left( \cancel{\Tr(\sigma_0 \sigma_k)} + r_j \Tr(\sigma_j \sigma_k) \right) = \inv{2} r_j 2 \delta_{jk} = r_k, \]
    por lo que en términos vectoriales $\expval{\vb{\sigma}} = \vb{r}$.
    
    
    
    
    %-------------------------------------------------------------------------------------------------------
    %   Problema I.5
    %-------------------------------------------------------------------------------------------------------
    \item Para un sistema de dos qubits, es natural pensar que base de representación en términos de las matrices de Pauli contendrá a los elementos $\mathds{1} = \mathds{1}_A \otimes \mathds{1}_B$, $\sigma_k \otimes \mathds{1}$ y $\mathds{1} \otimes \sigma_k$. Pero adicionalmente deberemos considerar también los términos $\sigma_j \otimes \sigma_k$. Por lo tanto, una base completa estará conformada por
    \[ B = \{ \mathds{1}, \sigma_x, \sigma_y, \sigma_z \} \otimes \{ \mathds{1}, \sigma_x, \sigma_y, \sigma_z \}. \]
    
    Un estado genérico para dos qubits podrá ser expresado como
    \[ \rho_{AB} = \alpha \mathds{1} + \vb{r}_A \cdot \vb{\sigma} \otimes \mathds{1} + \mathds{1} \otimes \vb{\sigma} \vdot \vb{r}_B + \sum_{i,j = 1}^3 r_{ij} \sigma_i \otimes \sigma_j. \]
    Como $\Tr{\rho_{AB}} = \alpha \Tr{\mathds{1}} = 4 \alpha = 1$ por la condición de normalización (al ser $\Tr{\sigma_k} = 0$), entonces $\alpha = 1/4$. Podemos sacar a $\alpha$ como factor común de todo el operador densidad, resultando en
    \[ \rho_{AB} = \inv{4} \left(\mathds{1} + \vb{r}_A \cdot \vb{\sigma} \otimes \mathds{1} + \mathds{1} \otimes \vb{\sigma} \vdot \vb{r}_B + \sum_{i,j = 1}^3 r_{ij} \sigma_i \otimes \sigma_j \right). \]
    
    Ahora, como
    \[ \Tr{(\sigma_i \otimes \sigma_j)(\sigma_{i'} \otimes \sigma_{j'})} = \Tr_A (\sigma_i \sigma_{i'}) \Tr_B (\sigma_j \sigma_{j'}) = 2 \delta_{i i'} \ 2 \delta_{j j'} = 4 \delta_{i i'} \delta_{j j'}, \]
    entonces
    \[ \expval{\vb{\sigma}_A} = \Tr(\rho_{AB} \vb{\sigma} \otimes \mathds{1}) = \vb{r}_A, \]
    \[ \expval{\vb{\sigma}_B} = \Tr(\rho_{AB} \mathds{1} \otimes \vb{\sigma}) = \vb{r}_B, \]
    y
    \[ \expval{\sigma_i \otimes \sigma_j} = \Tr(\rho_{AB} \sigma_i \otimes \sigma_j) = r_{ij}. \]
    
    
    
    %-------------------------------------------------------------------------------------------------------
    %   Problema I.6
    %-------------------------------------------------------------------------------------------------------
    \item Sea un operador densidad $\rho = x \dyad{\Phi}{\Phi} + (1-x) \mathds{1}_d / d$, siendo $B = \{ \ket{\Phi} \} \cup \{ \ket{\phi_{k}} : k = 2,\dots, d \}$ una base ortonormal del sistema de dimensión $d$. Luego, los valores posibles de $x$ estarán restringidos por la condición de normalización $\Tr \rho = 1$:
    \[ \Tr \rho = x \underbrace{\Tr(\dyad{\Phi}{\Phi})}_{=1} + \frac{1-x}{d} \underbrace{\Tr{\mathds{1}_d}}_{=d} = x + \frac{1-x}{d} d = x + 1 - x = 1. \]
    Por lo tanto, $\rho$ no presenta restricciones de $x$ por la condición de normalización, exceptuando $x \in \realSet$ para que se cumpla la hermiticidad del operador ($\rho^\dagger = \rho$).
    
    La condición de positividad de $\rho$ puede ser expresada dado un vector $\ket{\psi}$ arbitrario del espacio como
    \[ \mel{\psi}{\rho}{\psi} = x \abs{\ip{\psi}{\Phi}}^2 + \frac{1 - x}{d} \ip{\psi}{\psi} = x \left( \abs{\ip{\psi}{\Phi}}^2 - \inv{d} \right) + \inv{d} \geq 0. \]
    Eligiendo $\ket{\psi} = \ket{\Phi}$ (todavía no me convence, me parece que falta una hipótesis adicional, ya que se rompe eligiendo $\ket{\psi} = \alpha \ket{\Phi} + \beta_k \ket{k}$, para $\alpha < 1$, $\ket{k} \neq \ket{\Phi} \forall k$ y $\ip{\psi}{\psi} = 1$), entonces
    \[ x \left( \abs{\ip{\Phi}{\Phi}}^2 - \inv{d} \right) + \inv{d} = x \left( 1 - \inv{d} \right) + \inv{d} \geq 0 \iff x \geq -\inv{d \left(1 - \frac{1}{d} \right)} = \inv{1 - d}. \]
    
    Para $x=1$, se tratará de un estado puro ($\rho = \dyad{\Phi}{\Phi}$). Por el contrario, para $x = 0$, el sistema se encontrará en un estado máximamente entrelazado ($\rho = \mathds{1}_d / d$).
    
\end{enumerate}


\section{Estados de sistemas compuestos. Entrelazamiento.}
\begin{enumerate}
    
    %-------------------------------------------------------------------------------------------------------
    %   Problema II.1
    %-------------------------------------------------------------------------------------------------------
    \item Sea un sistema de dos qubits en la base computacional $\{ \ket{00}, \ket{01}, \ket{10}, \ket{11} \}$, con matrices densidad definidas por $\rho = \dyad{\Phi_{AB}}{\Phi_{AB}}$.
    \begin{enumerate}
        \item Dado $\ket{\Phi_{AB}} = \frac{\ket{00} \pm \ket{11}}{\sqrt{2}}$,
        \[ \implies \rho = \inv{2} \left( \ket{00} \pm \ket{11} \right) \left(\bra{00} \pm \bra{11} \right) = \inv{2} \left( \dyad{00}{00} \pm \dyad{00}{11} \pm \dyad{11}{00} + \dyad{11}{11} \right). \]
        
        En términos matriciales, identificando
        \[
            \ket{\Phi_{AB}} \equiv \inv{\sqrt{2}}
            \begin{pmatrix}
                1 \\ 0 \\ 0 \\ \pm 1
            \end{pmatrix},
        \]
        entonces
        \[
            \rho \equiv \inv{2}
            \begin{pmatrix}
                1 \\ 0 \\ 0 \\ \pm 1
            \end{pmatrix}
            \begin{pmatrix}
                1 & 0 & 0 & \pm 1
            \end{pmatrix}
            =
            \inv{2}
            \begin{pmatrix}
                1 & 0 & 0 & \pm 1 \\
                0 & 0 & 0 & 0 \\
                0 & 0 & 0 & 0 \\
                \pm 1 & 0 & 0 & 1
            \end{pmatrix}.
        \]
        Los autovalores correspondientes estarán dados en ambos casos por $\lambda \in (1, 0, 0, 0)$.
        
        
        \item Para
        \[
            \ket{\Phi_{AB}} = \frac{\ket{00} + \ket{10} - \ket{01} - \ket{11}}{2} \equiv \inv{2}
            \begin{pmatrix}
                1 \\ -1 \\ 1 \\ -1
            \end{pmatrix},
        \]
        por lo que
        \[
            \rho \equiv \inv{4}
            \begin{pmatrix}
                1 \\ -1 \\ 1 \\ -1
            \end{pmatrix}
            \begin{pmatrix}
                1 & -1 & 1 & -1
            \end{pmatrix}
            =
            \inv{4}
            \begin{pmatrix}
                1 & -1 & 1 & -1 \\
                -1 & 1 & -1 & 1 \\
                1 & -1 & 1 & -1 \\
                -1 & 1 & -1 & 1
            \end{pmatrix},
        \]
        con autovalores $\lambda \in (1, 0, 0, 0)$.
    \end{enumerate}
    
    
    
    %-------------------------------------------------------------------------------------------------------
    %   Problema II.2
    %-------------------------------------------------------------------------------------------------------
    \item Las matrices de densidad reducidas estarán dadas por $\rho_A = \Tr_B{\rho_{AB}}$.
    \begin{enumerate}
        \item Teniendo que
        \[ \rho_{AB} = \inv{2} \left( \dyad{00}{00} \pm \dyad{00}{11} \pm \dyad{11}{00} + \dyad{11}{11} \right), \]
        entonces
        \[ \rho_A = \Tr_B{\rho_{AB}} = \inv{2} \left( \dyad{0}{0} + \dyad{1}{1} \right). \]
        Luego, la entropía de entrelazamiento será
        \[ S(\rho_A) = -\Tr(\rho_A \log_2 \rho_A) = - \left( \inv{2^2} \log_2(\inv{2^2}) + \inv{2^2} \log_2(\inv{2^2}) \right) = \inv{2} 2 \log_2(2) = 1. \]
        
        \item En este caso,
        \[ \rho_{AB} = \inv{4} \left( \ket{00} + \ket{10} - \ket{01} - \ket{1} \right) \left( \bra{00} + \bra{10} - \bra{01} - \bra{1} \right). \]
        Entonces,
        \begin{align}
            \rho_A = \Tr_B{\rho_{AB}} &= \inv{4} \left( \dyad{0}{0} + \dyad{0}{1} + \dyad{1}{0} + \dyad{1}{1} + \dyad{0}{0} + \dyad{0}{1} + \dyad{1}{0} + \dyad{1}{1} \right) \\
            &= \inv{2} \left( \dyad{0}{0} + \dyad{0}{1} + \dyad{1}{0} + \dyad{1}{1} \right) \\
            &= \inv{2}
            \begin{pmatrix}
                1 & 1 \\
                1 & 1 \\
            \end{pmatrix}.
        \end{align}
        Luego, como $\rank \rho_A = n_S = 1$, entonces se trata de un estado puro, por lo que $S(\rho_A) = 0$.
    \end{enumerate}
    
    
    
    %-------------------------------------------------------------------------------------------------------
    %   Problema II.3
    %-------------------------------------------------------------------------------------------------------
    \item La descomposición de Schmidt será tal que $\ket{\psi_{AB}} = \sum_{k} \sigma_k^2 \ket{k}_A \ket{k}_B$, con $\sigma_k \geq 0$, obtenidos a partir de la SVD de la matriz $C$ definida como
    \[ \ket{\psi_{AB}} = \sum_{ij} C_{ij} \ket{i}_A \ket{j}_B. \]
    Mediante la SVD, escribiremos a $C$ como $C = U D V^\dagger$, con $U, V$ matrices unitarias y $D$ matriz diagonal de autovalores $\sigma_k$.
    \begin{enumerate}
        \item En este caso, $\ket{\Phi_{AB}^{\pm}} = \inv{\sqrt{2}} \left( \ket{00} \pm \ket{11} \right)$, el sistema ya se encuentra en la descomposición de Schmidt. Para cumplir con la condición adicional de $\sigma_k \geq 0$, notamos que
        \[ \ket{\Phi_{AB}^{+}} = \inv{\sqrt{2}} \left( \ket{00} + \ket{11} \right) = \inv{\sqrt{2}} \ket{0}_A \ket{0}_B + \inv{\sqrt{2}} \ket{1}_A \ket{1}_B = \sum_{k = 0}^{1} \sigma_k \ket{k}_A \ket{k}_B, \]
        con $\sigma_0 = \sigma_1 = 1/\sqrt{2}$, y estados en una base $\ket{k}_A \in \{ \ket{0}_A, \ket{1}_A \}$ y $\ket{k}_B \in \{ \ket{0}_B, \ket{1}_B \}$.
        Para el otro caso,
        \[ \ket{\Phi_{AB}^{-}} = \inv{\sqrt{2}} \left( \ket{00} - \ket{11} \right) = \inv{\sqrt{2}} \ket{0}_A \ket{0}_B + \inv{\sqrt{2}} (i\ket{1}_A)(i\ket{1}_B) = \sum_{k = 0}^{1} \sigma_k \ket{k}_A \ket{k}_B, \]
        siendo nuevamente $\sigma_0 = \sigma_1 = 1/\sqrt{2}$, pero esta vez con bases $\ket{k}_A \in \{ \ket{0}_A, i\ket{1}_A \}$ y $\ket{k}_B \in \{ \ket{0}_B, i\ket{1}_B \}$.
        
        
        \item Para $\ket{\Phi_{AB}} = \inv{2} \left(\ket{00} + \ket{10} - \ket{01} - \ket{11}\right)$, podemos construir la matriz $C_{ij}$ como
        \[
            C = \inv{2}
            \begin{pmatrix}
                1 & -1 \\
                1 & -1 
            \end{pmatrix}.
        \]
        Luego, evaluando la SVD en Mathematica, obtenemos $C = U D V^\dagger$
        \[
            U = \inv{\sqrt{2}}
            \begin{pmatrix}
                -1 & -1 \\
                -1 & 1
            \end{pmatrix},
            \quad \quad
            V = \inv{\sqrt{2}}
            \begin{pmatrix}
                -1 & 1 \\
                1 & 1
            \end{pmatrix},
            \quad \quad
            D = \inv{\sqrt{2}}
            \begin{pmatrix}
                1 & 0 \\
                0 & 0
            \end{pmatrix}.
        \]
        Los elementos de la nueva base estarán determinados como
        \[ \ket{k}_A = \sum_i U_{ik} \ket{i}_A \implies \ket{k = 0}_A = -\frac{\ket{0}_A + \ket{1}_A}{\sqrt{2}}, \quad \ket{k = 1}_A = \frac{-\ket{0}_A + \ket{1}_A}{\sqrt{2}} \]
        y
        \[ \ket{k}_B = \sum_j V_{jk}^* \ket{j}_B \implies \ket{k = 0}_B = \frac{-\ket{0}_B + \ket{1}_B}{\sqrt{2}}, \quad \ket{k = 1}_B = \frac{\ket{0}_B + \ket{1}_B}{\sqrt{2}}. \]
        Por lo tanto, en la base de Schmidt
        \[ \ket{\Phi_{AB}} = 1 \ket{k=0}_A \ket{k=0}_B. \]
        Reemplazando $\ket{k = 0}_A$ y $\ket{k = 0}_B$, podemos verificar el cálculo recuperando el resultado inicial:
        \begin{align}
            \ket{\Phi_{AB}} = \ket{k=0}_A \ket{k=0}_B &= \inv{2} \left( -\ket{0}_A - \ket{1}_A \right) \left( -\ket{0}_B + \ket{1}_B \right) \\
                &= \inv{2} \left( \ket{00} - \ket{01} + \ket{10} - \ket{11} \right).
        \end{align}
    \end{enumerate}
    
    
    
    %-------------------------------------------------------------------------------------------------------
    %   Problema II.4
    %-------------------------------------------------------------------------------------------------------
    \item Sea un estado $\ket{\Psi_{AB}} = \frac{\alpha}{\sqrt{2}} \left( \ket{00} + \ket{11} \right) + \frac{\beta}{\sqrt{2}} \left( \ket{01} + \ket{10} \right)$, con $\abs{\alpha}^2 + \abs{\beta}^2 = 1$. Su descomposición de Schmidt estará determinada por $C = U D V^\dagger$, con
    \[
        D = \inv{\sqrt{2}}
        \begin{pmatrix}
            \alpha & \beta \\
            \beta & \alpha \\
        \end{pmatrix}.
    \]
    Podemos determinar $V y D$ diagonalizando el producto
    \[ C^\dagger C = V D^\dagger U^\dagger U D V^\dagger = V D^\dagger D V^\dagger = V D^2 V^{-1}. \]
    En particular en este ejercicio,
    \begin{align}
        C^\dagger C = \inv{2}
        \begin{pmatrix}
            \alpha^* & \beta^* \\
            \beta^* & \alpha^*
        \end{pmatrix}
        \begin{pmatrix}
            \alpha & \beta \\
            \beta & \alpha
        \end{pmatrix}
        &= \inv{2}
        \begin{pmatrix}
            \abs{\alpha}^2 + \abs{\beta}^2 & \alpha^* \beta + \beta^* \alpha \\
            \alpha^* \beta + \beta^* \alpha & \abs{\alpha}^2 + \abs{\beta}^2
        \end{pmatrix} \\
        &= \inv{2}
        \begin{pmatrix}
            1 & \Gamma \\
            \Gamma & 1
        \end{pmatrix} \\
        &=
        \begin{pmatrix}
            -1 & 1 \\
            1 & 1
        \end{pmatrix}
        \begin{pmatrix}
            \frac{1-\Gamma}{2} & 0 \\
            0 & \frac{1 + \Gamma}{2}
        \end{pmatrix}
        \begin{pmatrix}
            -1 & 1 \\
            1 & 1
        \end{pmatrix}^{-1}.
    \end{align}
    \[
        \implies
        V =
        \begin{pmatrix}
            -1 & 1 \\
            1 & 1
        \end{pmatrix},
        \quad \quad
        D =
        \begin{pmatrix}
            \sqrt{\frac{1-\Gamma}{2}} & 0 \\
            0 & \sqrt{\frac{1 + \Gamma}{2}}
        \end{pmatrix}
    \]
    Así mismo, como $C C^\dagger = C^\dagger C$, entonces también $U = V$. Por lo tanto, tendremos
    \[ \sigma_1 = \sqrt{\frac{1-\Gamma}{2}}, \quad \sigma_2 = \sqrt{\frac{1+\Gamma}{2}}, \]
    y la base de Schmidt estará dada por
    \[ \mathcal{B}_A = \left\{ \ket{k=1}_A = -\ket{0}_A + \ket{1}_A, \ket{k=2}_A = \ket{0}_A + \ket{1}_A \right\} \]
    y
    \[ \mathcal{B}_B = \left\{ \ket{k=1}_B = -\ket{0}_B + \ket{1}_B, \ket{k=2}_B = \ket{0}_B + \ket{1}_B \right\}. \]
    
    \begin{enumerate}
        \item El estado será puro cuando
        \[ \Gamma = \alpha^* \beta + \beta^* \alpha = 2 \Re{\alpha \beta^*} = \pm 1. \]
        \item El estado será entrelazado cuando $-1 < \Gamma < 1$.
        \item El sistema presentará entrelazamiento máximo cuando
        \begin{align}
            S(\rho_A) &= - \Tr{\rho_A \log \rho_A} \\
                &= -\left[ \qty(\sqrt{\frac{1+\Gamma}{2}})^2 \log_2\qty(\qty(\sqrt{\frac{1+\Gamma}{2}})^2) + \qty(\sqrt{\frac{1-\Gamma}{2}})^2 \log_2\qty(\qty(\sqrt{\frac{1-\Gamma}{2}})^2) \right] \\
                &= -\left[ \frac{1+\Gamma}{2} \log_2\qty(1+\Gamma) + \frac{1-\Gamma}{2} \log_2\qty(1-\Gamma) - 1 \right]
        \end{align}
        sea máximo. Es decir,
        \[ \pdv{S(\rho_A)}{\Gamma} = \frac{1}{2} \left( \log_2(1-\Gamma) - \log_2(1+\Gamma) \right) = 0 \implies 1-\Gamma = 1+\Gamma \implies \Gamma = 0. \]
    \end{enumerate}
    
    
    
    %-------------------------------------------------------------------------------------------------------
    %   Problema II.5
    %-------------------------------------------------------------------------------------------------------
    \item Sea un sistema de 2 qubits en el estado de Bell $\ket{\Psi_{AB}} = \frac{\ket{01} + \ket{10}}{\sqrt{2}}$, cuya matriz densidad será
    \[ \rho_{AB} = \dyad{\Psi_{AB}}{\Psi_{AB}} = \inv{2} \left( \dyad{01}{01} + \dyad{10}{10} + \dyad{10}{01} + \dyad{01}{10} \right). \]
    Sea un segundo sistema en el estado representado por el operador densidad
    \[ \tilde{\rho}_{AB} = \inv{2} \left( \dyad{01}{01} + \dyad{10}{10} \right). \]
    Luego,
    \[ \rho_{AB} = \tilde{\rho}_{AB} + \inv{2} \left( \dyad{10}{01} + \dyad{01}{10} \right). \]
    Además, $\tilde{\rho}_{AB}$ no será un estado puro, ya que
    \[ \tilde{\rho}_{AB}^2 = \inv{4} \left( \dyad{01}{01} + \dyad{10}{10} \right) \left( \dyad{01}{01} + \dyad{10}{10} \right) = \inv{4} \left( \dyad{01}{01} + \dyad{10}{10} \right) = \frac{\tilde{\rho}_{AB}}{2} \neq \tilde{\rho}_{AB}. \]
    
    Será posible distinguir los estados mediante un observable global $\hat O = \hat O_A \otimes \hat O_B$, pero no así mediante un observable local $\hat O_A \otimes \mathds{1}_B$, ya que
    \[ \rho_A = \Tr_B(\rho_{AB}) = \inv{2} \left( \dyad{1}{1} + \dyad{0}{0} \right) =\Tr_B(\tilde{\rho}_{AB}) = \tilde{\rho}_A, \]
    y
    \[ \expval{\hat O_A}_A = \Tr(\hat O_A \rho_A) = \Tr(\hat O_A \tilde{\rho}_A) = \expval{\hat O_A}_{\tilde{A}}. \]
    
\end{enumerate}



%\nocite{*}
%\printbibliography
\end{document}