\documentclass[12pt]{article}
\textheight=280mm
\evensidemargin=-0cm
\oddsidemargin=-0cm
\textwidth=160mm
\topmargin=-2cm
\usepackage{bbm}
\newcommand{\be}{\begin{equation}}
\newcommand{\ee}{\end{equation}}
\newcommand{\ben}{\begin{eqnarray}}
\newcommand{\een}{\end{eqnarray}}
\newcommand{\half}{\mbox{${\textstyle\frac{1}{2}}$}}
\newcommand{\quart}{\mbox{${\textstyle\frac{1}{4}}$}}
\newcommand{\sen}{\mbox{${\rm sen}$}}
\newcommand{\senh}{\mbox{${\rm senh}$}}
\newcommand{\arcsen}{\mbox{${\rm arcsen}$}}
\newcommand{\dxy}{\mbox{$\frac{dx}{dy}$}}
\newcommand{\dyx}{\mbox{$\frac{dy}{dx}$}}
\newcommand{\rdo}{\mbox{${\Rightarrow}$}}
%\newcommand{\arccos}{\mbox{${\rm arccos}$}}
%\newcommand{\arctan}{\mbox{${\rm arctan}$}}
%\newcommand{\cos}{\mbox{${\rm cos}$}}
%\newcommand{\tan}{\mbox{${\rm tan}$}}
\newcommand{\pid}{\mbox{${\textstyle\frac{\pi}{2}}$}}
\newcommand{\pic}{\mbox{${\textstyle\frac{\pi}{4}}$}}
\newcommand{\pitt}{\mbox{${\textstyle\frac{\pi}{3}}$}}
\newcommand{\pis}{\mbox{${\textstyle\frac{\pi}{6}}$}}
%\newcommand{\quart}{\mbox{$\case{1}{4}$}}
\renewcommand{\thepage}{-- \arabic{page} --}
\begin{document}
\begin{center}
\subsection*{Seminario de Mec\'anica Cu\'antica / \\Teor\'{\i}a de la Informaci\'on Cu\'antica}
\subsection*{Pr\'actica VI (Curso 2020)}
\end{center}
I. {\bf Estados coherentes.}\\ 
1) Sea 	
\[|\alpha\rangle=e^{-|\alpha|^2/2}\sum_{n=0}^\infty\frac{\alpha^n}{\sqrt{n!}}|n\rangle\] 
un estado coherente, donde $|n\rangle=(a^\dagger)^n|0\rangle/\sqrt{n!}$ y $a^\dagger,a$ son operadores de creaci\'on y aniquilaci\'on bos\'onicos. Demostrar las siguientes propiedades:\hfill\break \\
a) $|\alpha\rangle=e^{-|\alpha|^2/2}\exp(\alpha a^\dagger)|0\rangle$\hfill\break
b) $|\alpha\rangle=T(\alpha)|0\rangle$, donde $T(\alpha)=\exp[\alpha a^\dagger-\alpha^*a]
=\exp[-i\sqrt{2}({\rm Re}(\alpha) p-{\rm Im}(\alpha) q)]$ es un operador de 
traslaci\'on y $p=\frac{a-a^\dagger}{\sqrt{2}i}$, $q=\frac{a^\dagger+a}{\sqrt{2}}$ 
son los operadores impulso y coordenada asociados. Esto  muestra que $|\alpha\rangle$ es 
un estado fundamental de oscilador arm\'onico ``trasladado''. Notar que $T(\alpha)$ es unitario.\hfill\break
c)  $a|\alpha\rangle=\alpha|\alpha\rangle$ (autoestado del operador aniquilaci\'on). \hfill\break
d) $P(n)=|\langle n|\alpha\rangle|^2$ sigue una distribuci\'on de Poisson, 
con $E(n)=\langle \alpha|a^\dagger a|\alpha\rangle=|\alpha|^2=V(n)$. 
%$V(n)=\langle \hat{n}^2\rangle-\langle \hat{n}\rangle^2=|\alpha|^2$. 
\hfill\break 
e) $\langle\alpha|p|\alpha\rangle=\sqrt{2}{\rm Im}(\alpha)$,  
$\langle\alpha|q|\alpha\rangle=\sqrt{2}{\rm Re}(\alpha)$\,.\hfill\break 
f) Si $H=\hbar\omega a^\dagger a$ $\Rightarrow$ $\exp[-iHt]|\alpha\rangle=|\alpha(t)\rangle$, 
con $\alpha(t)=e^{-i\omega t}\alpha$.\hfill\break La evoluci\'on temporal de $|\alpha\rangle$ 
es pues una rotaci\'on del par\'ametro complejo $\alpha$, con frecuencia angular
 $\omega$.
\hfill\break
e) No ortogonalidad: $\langle\beta|\alpha\rangle=e^{-(|\alpha^2|+|\beta|^2)/2+\beta^*\alpha}$\hfill\break
h) (Sobre)completitud: $\frac{1}{\pi}\int_{C}d\alpha|\alpha\rangle\langle\alpha|=I$ (Identidad).\hfill\break 
%%<np|\alpha><\alpha|n>=e^{-|\alpha|^2} \alphac^np\alpha^n/\sqrt{np!n!}=e^{-r^2}r^(2n+1)(x-iy)^(np-n)dr dtheta
%=e^{-x^2-y^2}(x-iy)^np(x+iy)^n=(x^2+y^2)^n(x-iy)^(np-n)=e^{-r^2}r^{2n} 
i) Incerteza M\'{\i}nima: $\Delta p\Delta q=1/2$, 
donde $(\Delta p)^2=\langle p^2\rangle-\langle p\rangle^2$, 
$(\Delta q)^2=\langle q^2\rangle-\langle q\rangle^2$, 
con los valores medios tomados respecto 
de $|\alpha\rangle$. 
Los estados coherentes 
son pues los estados cu\'anticos m\'as ``cercanos'' a estados 
cl\'asicos de oscilador arm\'onico. \\ \\
2) a) Mostrar que el efecto de un divisor de haces (beamsplitter), representado por el operador  
unitario 
$U=\exp[-i\theta(a_1a^\dagger_2+a^\dagger_1a_2)]$, sobre un estado coherente es 
\[U|\alpha\rangle\otimes|\beta\rangle=|\alpha\cos\theta -i\beta\sin\theta\rangle
\otimes|\beta\cos\theta-i\alpha\sin\theta\rangle\]
b) Hallar $U|10\rangle$, $U|01\rangle$, donde $|10\rangle=a^\dagger_2|00\rangle$, 
$|01\rangle=a^\dagger_1|00\rangle$. \\\hfill\break
% y pueden ser en principio tambi\'en utilizados 
%para representar qubits.
%\end{document}
 \hfill\break
 {\bf II.  Transformaciones unitarias y de Bogoliubov.}  \\
1) Hallar las energ\'ias, autoestados y el  estado fundamental de los siguientes Hamiltonianos:\\
%a) (dos qubits) 
%\[ H=B(\sigma_z^A+\sigma_z^B)-J(\sigma_+^A\sigma_-^B+\sigma_-^A\sigma_+^B)\]
%Considerar los casos i) $0<J<B$ y ii) $0<B<J$. \\
a) %(dos osciladores cu\'anticos acoplados, $a^\dagger$, $a$ operadores de creaci\'on y aniquilaci\'on bos\'onicos)
\[H=\varepsilon(c^\dagger_1 c_1+c^\dagger_2 c_2)-v(c^\dagger_1 c_2+c^\dagger_2 c_1)\]
b) %(dos osciladores o cu\'anticos acoplados)
\[H=\varepsilon(c^\dagger_1 c_1+c^\dagger_2 c_2)-v(c^\dagger_1 c^\dagger_2+c_2 c_1),
\]
Considerar tanto el caso fermi\'onico como el bos\'onico (en este caso para $|v|<\varepsilon$; Justificar esta restricci\'on).  
\\
2) Hallar el entrelazamiento (de los modos 1 y 2) del estado fundamental en a)--b) para $|g|<\varepsilon$. 
\end{document}

