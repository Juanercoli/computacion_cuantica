\documentclass{scrartcl}
\usepackage[a4paper,margin=2cm,footskip=1cm]{geometry}
\setkomafont{disposition}{\normalfont\bfseries}

%\documentclass{article}

\usepackage[table,xcdraw]{xcolor}
\usepackage{tikz}
\usetikzlibrary{angles,quotes}
\usetikzlibrary{babel}

\usepackage{booktabs}
\usepackage[utf8]{inputenc}
\usepackage[spanish, es-nodecimaldot]{babel}
\usepackage[per-mode=symbol]{siunitx}
\usepackage{graphicx}
\usepackage{subcaption}
\usepackage{caption}
\usepackage{mathtools}
\usepackage{amsmath}
\usepackage{slashed}
\usepackage{dsfont}
\usepackage{float}
\usepackage{multicol}
\usepackage{wrapfig}
\usepackage{lipsum}
\usepackage{textcomp}
\usepackage{gensymb}
\usepackage{longtable}
\usepackage{supertabular}
\usepackage{hhline}
\usepackage{enumerate}
\usepackage{multirow}
\usepackage{amssymb}
\usepackage{tabularx}
\usepackage{ragged2e}
\usepackage{rotating}
\usepackage{cancel}
\usepackage{physics}
\usepackage[framemethod=default]{mdframed}
\usepackage{csquotes}
%\usepackage[backend=biber, style=numeric, sorting=none]{biblatex}
\usepackage{qcircuit}

\renewcommand{\figurename}{Figura}
\renewcommand{\spanishtablename}{Tabla}
\newcommand{\inv}[1]{\frac{1}{#1}}
\newcommand{\uv}[1]{\hat{\mathbf{#1}}}
\newcommand{\uvs}[1]{\, \uv{#1}}

\newcommand{\realSet}{\mathbb{R}}
\newcommand{\complexSet}{\mathbb{C}}
\newcommand{\oref}{$\mathcal{O}$ }
\newcommand{\opref}{$\mathcal{O}'$ }
\newcommand{\oppref}{$\mathcal{O}''$ }

\def\residue{\mathop{\text{Res}}}

\setlength{\tabcolsep}{19pt}

\DeclareSIUnit\clight{\text{\ensuremath{c}}}
\DeclareSIUnit\MeV{\mega\electronvolt}
\DeclareSIUnit\GeV{\giga\electronvolt}
\DeclareSIUnit\MeVpc{\MeV\per\clight\squared}
\DeclareSIUnit\GeVpc{\GeV\per\clight\squared}

%\newcommand{\vbeta}{\vb{\beta}}
\newcommand{\sinc}{\text{sinc}}
\newcommand{\E}{\vb{E}}
\newcommand{\B}{\vb{B}}
\newcommand{\x}{\vb{x}}
\newcommand{\y}{\vb{y}}
\newcommand{\z}{\vb{z}}
\newcommand{\p}{\vb{p}}
\renewcommand{\k}{\vb{k}}
\newcommand{\Lag}{\mathcal{L}}
\newcommand{\Ham}{\mathcal{H}}

\newcommand{\tx}{\tilde{x}}

\renewcommand{\a}[1]{\hat{a}_{#1}}
\newcommand{\ad}[1]{\hat{a}_{#1}^\dagger}

\DeclareRobustCommand{\[}{\begin{equation}}
\DeclareRobustCommand{\]}{\end{equation}}
\mathtoolsset{showonlyrefs}

\allowdisplaybreaks

%\bibliography{bibliography}

%----------------------------------------------------------------------------------------
%	DOCUMENT INFORMATION
%----------------------------------------------------------------------------------------

\title{Teoría de la Información Cuántica}
\subtitle{Práctica 2 - Año 2020}
\author{\textsc{Beaucamp}, Jean Yves}
\date{}

\begin{document}

\maketitle

\section{Traza Parcial, Entrelazamiento y Medidas}
\begin{enumerate}
    
    %-------------------------------------------------------------------------------------------------------
    %   Problema I.1
    %-------------------------------------------------------------------------------------------------------
    \item Consideraremos al sistema de $n$ qubits dividido en dos subsistemas dados por una partición $(m, n - m)$, con $1 \leq m \leq n -1$.
    \begin{enumerate}[(I)]
        \item $\displaystyle \ket{\Phi} = \inv{\sqrt{2}} \left( \ket{0 \dots 0} + \ket{1 \dots 1} \right)$. Introducimos la partición del sistema como
        \begin{align}
            \ket{\Phi} &= \inv{\sqrt{2}} \left( \ket{0_1 \dots 0_m} \ket{0_{m + 1} \dots 0_n} + \ket{1_1 \dots 1_m} \ket{1_{m + 1} \dots 1_n} \right) \\
            &= \inv{\sqrt{2}} \left( \ket{0_m} \ket{0_{n - m}} + \ket{1_m} \ket{1_{n-m}} \right)
        \end{align}
        La matriz densidad estará dada por
        \begin{align}
            \rho &= \dyad{\Phi}{\Phi} \\
                &= \inv{2} \left( \ket{0_m} \ket{0_{n - m}} \bra{0_m} \bra{0_{n - m}} + \ket{0_m} \ket{0_{n - m}} \bra{1_m} \bra{1_{n - m}} \right. \\
                &\quad \quad \left. + \ket{1_m} \ket{1_{n - m}} \bra{0_m} \bra{0_{n - m}} + \ket{1_m} \ket{1_{n - m}} \bra{1_m} \bra{1_{n - m}} \right).
        \end{align}
        Luego, el estado reducido será
        \[ \rho_m = \Tr_{n-m} \rho = \inv{2} \left( \ket{0_m} \bra{0_m} + \ket{1_m} \bra{1_m} \right), \]
        con una entropía de entrelazamiento (estando ya en la base de Schmidth)
        \[ E(\rho) = S(\rho_m) = -\sum_k \sigma_k^2 \log_2 \sigma_k^2 = - \left( \inv{2} \log_2 \inv{2} + \inv{2} \log_2 \inv{2} \right) = \inv{2} + \inv{2} = 1. \]
        Se trata, por lo tanto, del estado máximamente entrelazado.
        
        
        \item $\displaystyle \ket{\Phi} = \inv{\sqrt{n}} \left( \ket{1 0 \dots 0} + \ket{01 \dots 0} + \dots + \ket{0 \dots 01} \right)$. Introducimos la partición del sistema como
        \begin{align}
            \ket{\Phi} &= \inv{\sqrt{n}} \left( \ket{1_1 0_2 \dots 0_m} \ket{0_{m + 1} \dots 0_n} + \ket{0_1 1_2 \dots 0_m} \ket{0_{m + 1} \dots 0_n} \right. \\
            &\quad \quad \left. + \dots + \ket{0_1 0_2 \dots 1_m} \ket{0_{m + 1} \dots 0_n} + \right. \\
            &\quad \quad \left. + \ket{0_1 \dots 0_m} \ket{1_{m+1} 0_{m+2} \dots 0_n} + \dots + \ket{0_1 \dots 0_m} \ket{0_{m+1} 0_{m+2} \dots 1_n} \right) \\
            &= \inv{\sqrt{n}} \left\{ \sqrt{m} \left(\frac{\ket{1_1 0_2 \dots 0_m} + \ket{0_1 1_2 \dots 0_m} + \dots + \ket{0_1 0_2 \dots 1_m}}{\sqrt{m}} \right) \ket{0_{m + 1} \dots 0_n}  \right. \\
            &\quad \quad \left. + \sqrt{n - m} \ket{0_1 \dots 0_m} \left( \frac{\ket{1_{m+1} 0_{m+2} \dots 0_n} + \dots + \ket{0_{m+1} 0_{m+2} \dots 1_n}}{\sqrt{n - m}} \right) \right\} \\
            &= \sqrt{\frac{m}{n}} \ket{\phi_m} \ket{0_{n-m}} + \sqrt{\frac{n-m}{n}} \ket{0_m} \ket{\phi_{n-m}}.
        \end{align}
        La matriz densidad estará dada por
        \begin{align}
            \rho &= \dyad{\Phi}{\Phi} \\
                &= \frac{m}{n} \ket{\phi_m} \ket{0_{n-m}} \bra{\phi_m} \bra{0_{n-m}} + \frac{\sqrt{m(n-m)}}{n} \ket{0_m} \ket{\phi_{n-m}} \bra{\phi_m} \bra{0_{n-m}} \\
                &\quad \quad + \frac{\sqrt{m(n-m)}}{n} \ket{\phi_m} \ket{0_{n-m}} \bra{0_m} \bra{\phi_{n-m}} + \frac{(n-m)}{n} \ket{0_m} \ket{\phi_{n-m}} \bra{0_m} \bra{\phi_{n-m}}/
        \end{align}
        Luego, el estado reducido será
        \[ \rho_m = \Tr_{n-m} \rho = \frac{m}{n} \ket{\phi_m} \bra{\phi_m} + \frac{n-m}{n} \ket{0_m} \bra{0_m}, \]
        con una entropía de entrelazamiento
        \[ E(\rho) = S(\rho_m) = -\sum_k \sigma_k^2 \log_2 \sigma_k^2 = - \left( \frac{m}{n} \log_2 \frac{m}{n} + \frac{n-m}{n} \log_2 \frac{n-m}{n} \right). \]
        
    \end{enumerate}
    
    
    
    %-------------------------------------------------------------------------------------------------------
    %   Problema I.2
    %-------------------------------------------------------------------------------------------------------
    \item Dado el estado $\ket{\Psi_{AB}} = \sum_{ij} C_{ij} \ket{ij}$, con $\ip{ij}{kl} = \delta_{ik} \delta_{jl}$, y $\ket{ij} = \ket{i}_A \otimes \ket{j}_B$, su matriz densidad estará dada por
    \[ \rho_{AB} = \dyad{\Psi_{AB}}{\Psi_{AB}} = \sum_{ijkl} C_{ij} C_{kl}^* \dyad{ij}{kl}. \]
    \begin{enumerate}
        \item El estado reducido del subsistema A será
        \begin{align}
            \rho_A = \Tr_B \rho_{AB} = \sum_{m} \sum_{ijkl} C_{ij} C_{kl}^* \dyad{i}{k}_A \bra{m}\ket{j}\bra{l}\ket{m} &= \sum_{ijklm} C_{ij} C_{kl}^* \dyad{i}{k}_A \delta_{mj} \delta_{lm} \\
                &= \sum_{ijk} C_{ij} C_{kj}^* \dyad{i}{k}_A \\
                &= \sum_{ijk} C_{ij} C_{jk}^\dagger \dyad{i}{k}_A \\
                &= \sum_{ik} (CC^\dagger)_{ik} \dyad{i}{k}_A.
        \end{align}
        
        
        \item Realizando una medición local en $B$ con operadores de medida $M_m = \dyad{m}{m}_B$, el estado pos-medido del sistema estará dado por
        \[ \rho^{(m)}_{AB} = \frac{M_m \rho_{AB} M_m^\dagger}{\Tr M_m \rho_{AB} M_m^\dagger}. \]
        Evaluaremos primero el producto de operadores en el numerador.
        \begin{align}
            M_m \rho_{AB} M_m^\dagger &= \sum_{ijkl} C_{ij} C_{lk}^\dagger \dyad{i}{k}_A \otimes \ket{m}\bra{m} \ket{j}\bra{l} \ket{m}\bra{m}_B \\
                &= \sum_{ijkl} C_{ij} C_{lk}^\dagger \dyad{i}{k}_A \otimes \delta_{mj} \delta_{lm} \dyad{m}{m}_B \\
                &= \sum_{ik} C_{im} C_{mk}^\dagger \dyad{i}{k}_A \otimes \dyad{m}{m}_B.
        \end{align}
        Luego, la traza de este producto será
        \begin{align}
            \Tr M_m \rho_{AB} M_m^\dagger &= \sum_{pq} \sum_{ik} C_{im} C_{mk}^\dagger \bra{p}\ket{i}\bra{k}\ket{p} \bra{q}\ket{m}\bra{m}\ket{q} \\
                &= \sum_{pq} \sum_{ik} C_{im} C_{mk}^\dagger \delta_{ip} \delta_{kp} \delta_{qm} \\
                &= \sum_{p} C_{pm} C_{mp}^\dagger \\
                &= \sum_{p} C_{mp}^\dagger C_{pm} \\
                &= (C^\dagger C)_{mm}.
        \end{align}
        El estado pos-medido entonces corresponderá a la matriz densidad
        \[ \rho^{(m)}_{AB} = \frac{M_m \rho_{AB} M_m^\dagger}{\Tr M_m \rho_{AB} M_m^\dagger} = \sum_{ik} \frac{C_{im} C_{mk}^\dagger}{(C^\dagger C)_{mm}} \dyad{i}{k}_A \otimes \dyad{m}{m}_B. \]
        Luego, el estado reducido del subsistema $A$ será trivialmente
        \[ \rho^{(m)}_A = \Tr_B \rho^{(m)}_{AB} = \sum_{ik} \frac{C_{im} C_{mk}^\dagger}{(C^\dagger C)_{mm}} \dyad{i}{k}_A. \]
        
        
        \item El estado promedio luego de la medición será
        \begin{align}
            \rho^{(avg)}_{AB} = \sum_m P(m) \rho^{(m)}_{AB} &= \sum_m \Tr(M_m \rho_{AB} M_m^\dagger)  \rho^{(m)}_{AB} \\
                &= \sum_m \cancel{(C^\dagger C)_{mm}} \sum_{ik} \frac{C_{im} C_{mk}^\dagger}{\cancel{(C^\dagger C)_{mm}}} \dyad{i}{k}_A \otimes \dyad{m}{m}_B \\
                &= \sum_{ikm} C_{im} C_{mk}^\dagger \dyad{i}{k}_A \otimes \dyad{m}{m}_B.
        \end{align}
        Para el subsistema $A$ tendremos
        \begin{align}
            \rho^{(avg)}_{A} = \Tr_B \rho^{(avg)}_{AB} = \sum_l \sum_{ikm} C_{im} C_{mk}^\dagger \dyad{i}{k}_A \bra{l}\ket{m}\bra{m}\ket{l} &= \sum_{ikm} C_{im} C_{mk}^\dagger \dyad{i}{k}_A \\
                &= \sum_{ikm} (C C^\dagger)_{ik} \dyad{i}{k}_A.
        \end{align}
        
    \end{enumerate}
    
    
    
    %-------------------------------------------------------------------------------------------------------
    %   Problema I.3
    %-------------------------------------------------------------------------------------------------------
    \item Buscamos determinar $M_3$ y el valor máximo de $p$ tal que el conjunto $\{ M_1, M_2, M_3 \}$, con $M_1 = \sqrt{p} \dyad{1}{1}$, $M_2 = \sqrt{p} \dyad{-}{-}$, y $\ket{-} = \frac{\ket{0} - \ket{1}}{\sqrt{2}}$ represente una medida generalizada de un qubit. Sabiendo que $M_1^\dagger M_1 + M_2^\dagger M_2 + M_3^\dagger M_3 = \mathds{1}$:
    \[
        M_3^\dagger M_3 = \mathds{1} - M_1^\dagger M_1 - M_2^\dagger M_2 =
        \begin{pmatrix}
            1 & 0 \\
            0 & 1
        \end{pmatrix}
        -
        \begin{pmatrix}
            0 & 0 \\
            0 & p
        \end{pmatrix}
        -
        \begin{pmatrix}
            \frac{p}{2} & -\frac{p}{2} \\
            -\frac{p}{2} & \frac{p}{2}
        \end{pmatrix}
        =
        \begin{pmatrix}
            1-\frac{p}{2} & \frac{p}{2} \\
            \frac{p}{2} & 1-\frac{3p}{2}
        \end{pmatrix}.
    \]
    
    Como $P(j) = \lambda_j = \mel{\psi}{M_j^\dagger M_j}{\psi} = \norm{M_j \ket{\psi}}^2 \geq 0$, entonces los autovalores $\lambda_j$ de la matriz $M_j^\dagger M_j$ deberán ser no negativos. Por lo tanto, $\abs{M_j^\dagger M_j} > 0$. Estudiando en particular el caso para $M_3^\dagger M_3$:
    \[
        \abs{
        \begin{pmatrix}
            1-\frac{p}{2} & \frac{p}{2} \\
            \frac{p}{2} & 1-\frac{3p}{2}
        \end{pmatrix}
        } = \qty(1 - \frac{p}{2}) \qty(1 - \frac{3p}{2}) - \qty(\frac{p}{2})^2 = \qty(p - (2 + \sqrt{2})) \qty(p - (2 - \sqrt{2})) \geq 0.
    \]
    Luego, si queremos $p$ acotado, la condición de no-negatividad resulta
    \[ p - (2 \pm \sqrt{2}) \leq 0 \implies p \leq 2 - \sqrt{2}. \]
    
    Podemos expresar de forma genérica al conjunto de medida como $p \dyad{1}{1} + p \dyad{-}{-} + q \dyad{\psi}{\psi} = \mathds{1}$, con
    \[ \ket{\psi} = \cos \frac{\theta}{2} \ket{0} + \sin \frac{\theta}{2} \ket{1} \implies \dyad{\psi}{\psi} = 
    \begin{pmatrix}
        \cos^2\theta & \sin\theta \cos\theta \\
        \sin\theta \cos\theta & \sin^2\theta
    \end{pmatrix}.
    \]
    Luego,
    \[
    p
    \begin{pmatrix}
        \inv{2} & -\inv{2} \\
        -\inv{2} & \frac{3}{2}
    \end{pmatrix}
    + q
    \begin{pmatrix}
        \cos^2\theta & \sin\theta \cos\theta \\
        \sin\theta \cos\theta & \sin^2\theta
    \end{pmatrix}
    =
    \begin{pmatrix}
        1 & 0 \\
        0 & 1
    \end{pmatrix}
    \implies
    \begin{cases}
        \frac{p}{2} + q \cos^2\theta = 1 \\
        \frac{3p}{2} + q \sin^2\theta = 1
    \end{cases}
    \implies 2p + q = 2.
    \]
    
    Queremos ahora estudiar la capacidad de distinguir los estados $\ket{0}$ y $\ket{+}$ en el estado
    \[ \rho = \inv{2} \dyad{0}{0} + \inv{2} \dyad{+}{+}. \]
    
    Sobre el estado $\rho$,
    \[
        P(M_1) = \Tr(M_1^\dagger M_1 \rho) =
        \Tr(\begin{pmatrix}
            0 & 0 \\
            0 & p
        \end{pmatrix}
        \begin{pmatrix}
            \frac{3}{4} & \inv{4} \\
            \inv{4} & \inv{4}
        \end{pmatrix})
        =
        \Tr(\begin{pmatrix}
            0 & 0 \\
            \frac{p}{4} & \frac{p}{4}
        \end{pmatrix})
        = \frac{p}{4} \geq 1,
    \]
    y
    \[
        P(M_2) = \Tr(M_2^\dagger M_2 \rho) =
        \Tr(\frac{p}{2} \begin{pmatrix}
            1 & 1 \\
            1 & -1
        \end{pmatrix}
        \begin{pmatrix}
            \frac{3}{4} & \inv{4} \\
            \inv{4} & \inv{4}
        \end{pmatrix})
        =
        \frac{p}{2}
        \Tr(\begin{pmatrix}
            \frac{1}{2} & 0 \\
            -\frac{1}{2} & 0
        \end{pmatrix})
        = \frac{p}{4} \geq 1.
    \]
    Además, como $M_3$ es combinación de ambos, $\dyad{0}{0}$ y $\dyad{+}{+}$, entonces $P(3)$ no distingue entre $\dyad{0}{0}$ y $\dyad{+}{+}$.
    
\end{enumerate}

\section{Compuertas Lógicas Cuánticas}
\begin{enumerate}
    %-------------------------------------------------------------------------------------------------------
    %   Problema II.1
    %-------------------------------------------------------------------------------------------------------
    \item Sean $X = \sigma_x$, $Y = \sigma_y$, $Z = \sigma_z$. Luego,
    \begin{enumerate}
        \item
        \[
            X \otimes \mathds{1} =
            \begin{pmatrix}
                \mqty{\pmat{1}}
            \end{pmatrix}
            \otimes
            \begin{pmatrix}
                \mqty{\pmat{0}}
            \end{pmatrix}
            =
            \begin{pmatrix}
                \mqty{\mqty{\zmat{2}{2}} & \mqty{\pmat{0}} \\ \mqty{\pmat{0}} & \mqty{\zmat{2}{2}}}
            \end{pmatrix}.
        \]
        
        
        \item
        \[
            \mathds{1} \otimes X =
            \begin{pmatrix}
                \mqty{\pmat{0}}
            \end{pmatrix}
            \otimes
            \begin{pmatrix}
                \mqty{\pmat{1}}
            \end{pmatrix}
            =
            \begin{pmatrix}
                \mqty{\mqty{\pmat{1}} & \mqty{\zmat{2}{2}} \\ \mqty{\zmat{2}{2}} & \mqty{\pmat{1}}}
            \end{pmatrix}.
        \]
        
        
        \item
        \[
            X \otimes X =
            \begin{pmatrix}
                \mqty{\pmat{1}}
            \end{pmatrix}
            \otimes
            \begin{pmatrix}
                \mqty{\pmat{1}}
            \end{pmatrix}
            =
            \begin{pmatrix}
                \mqty{\mqty{\zmat{2}{2}} & \mqty{\pmat{1}} \\ \mqty{\pmat{1}} & \mqty{\zmat{2}{2}}}
            \end{pmatrix}.
        \]
        
        
        \item
        \begin{align}
            U_X = \dyad{0}{0} \otimes \mathds{1} + \dyad{1}{1} \otimes X &=
            \begin{pmatrix}
                1 & 0 \\
                0 & 0
            \end{pmatrix}
            \otimes
            \begin{pmatrix}
                \mqty{\pmat{0}}
            \end{pmatrix}
            +
            \begin{pmatrix}
                0 & 0 \\
                0 & 1
            \end{pmatrix}
            \otimes
            \begin{pmatrix}
                \mqty{\pmat{1}}
            \end{pmatrix} \\
            &=
            \begin{pmatrix}
                \mqty{\mqty{\pmat{0}} & \mqty{\zmat{2}{2}} \\ \mqty{\zmat{2}{2}} & \mqty{\zmat{2}{2}}}
            \end{pmatrix}
            +
            \begin{pmatrix}
                \mqty{\mqty{\zmat{2}{2}} & \mqty{\zmat{2}{2}} \\ \mqty{\zmat{2}{2}} & \mqty{\pmat{1}}}
            \end{pmatrix} \\
            &=
            \begin{pmatrix}
                \mqty{\mqty{\pmat{0}} & \mqty{\zmat{2}{2}} \\ \mqty{\zmat{2}{2}} & \mqty{\pmat{1}}}
            \end{pmatrix}.
        \end{align}
        
        Es trivial ver que $U_X \ket{00} = \ket{00}$, $U_X \ket{01} = \ket{01}$, $U_X \ket{10} = \ket{11}$ y $U_X \ket{11} = \ket{10}$, por lo que $U_X$ corresponde a la compuesta lógica cuántica \textit{Control Not}.
    \end{enumerate}
    
    En todos los casos resulta trivial ver que $U^t U = U^\dagger U = \mathds{1}_4$, por lo que las matrices de operadores lógicos son unitarias.
    
    
    
    %-------------------------------------------------------------------------------------------------------
    %   Problema II.2
    %-------------------------------------------------------------------------------------------------------
    \item Sea $H = \frac{X + Z}{\sqrt{2}} = \frac{1}{\sqrt{2}}\begin{pmatrix} 1 & 1 \\ 1 & -1 \end{pmatrix}$ la compuerta de Hadamard, y sea $W = U_X (H \otimes \mathds{1})$. Matricialmente,
    \begin{align}
        W &= U_X (H \otimes \mathds{1}) =
            \begin{pmatrix}
                \mqty{\mqty{\pmat{0}} & \mqty{\zmat{2}{2}} \\ \mqty{\zmat{2}{2}} & \mqty{\pmat{1}}}
            \end{pmatrix}
            \inv{\sqrt{2}}
            \begin{pmatrix}
                1 & 0 & 1 & 0 \\
                0 & 1 & 0 & 1 \\
                1 & 0 & -1 & 0 \\
                0 & 1 & 0 & -1
            \end{pmatrix}
            =
            \inv{\sqrt{2}}
            \begin{pmatrix}
                1 & 0 & 1 & 0 \\
                0 & 1 & 0 & 1 \\
                0 & 1 & 0 & -1 \\
                1 & 0 & -1 & 0
            \end{pmatrix}.
    \end{align}
    Luego, para estados en la base computacional
    \[
        W \ket{00} \equiv
        \inv{\sqrt{2}}
        \begin{pmatrix}
            1 & 0 & 1 & 0 \\
            0 & 1 & 0 & 1 \\
            0 & 1 & 0 & -1 \\
            1 & 0 & -1 & 0
        \end{pmatrix}
        \begin{pmatrix} 1 \\ 0 \\ 0 \\ 0 \end{pmatrix}
        = \inv{\sqrt{2}}
        \begin{pmatrix} 1 \\ 0 \\ 0 \\ 1 \end{pmatrix}
        \equiv \frac{\ket{00} + \ket{11}}{\sqrt{2}} = \ket{\Phi_+},
    \]
    \[
        W \ket{01} \equiv
        \inv{\sqrt{2}}
        \begin{pmatrix}
            1 & 0 & 1 & 0 \\
            0 & 1 & 0 & 1 \\
            0 & 1 & 0 & -1 \\
            1 & 0 & -1 & 0
        \end{pmatrix}
        \begin{pmatrix} 0 \\ 1 \\ 0 \\ 0 \end{pmatrix}
        = \inv{\sqrt{2}}
        \begin{pmatrix} 0 \\ 1 \\ 1 \\ 0 \end{pmatrix}
        \equiv \frac{\ket{01} + \ket{10}}{\sqrt{2}} = \ket{\Psi_+},
    \]
    \[
        W \ket{10} \equiv
        \inv{\sqrt{2}}
        \begin{pmatrix}
            1 & 0 & 1 & 0 \\
            0 & 1 & 0 & 1 \\
            0 & 1 & 0 & -1 \\
            1 & 0 & -1 & 0
        \end{pmatrix}
        \begin{pmatrix} 0 \\ 0 \\ 1 \\ 0 \end{pmatrix}
        = \inv{\sqrt{2}}
        \begin{pmatrix} 1 \\ 0 \\ 0 \\ -1 \end{pmatrix}
        \equiv \frac{\ket{00} - \ket{11}}{\sqrt{2}} = \ket{\Phi_-},
    \]
    \[
        W \ket{11} \equiv
        \inv{\sqrt{2}}
        \begin{pmatrix}
            1 & 0 & 1 & 0 \\
            0 & 1 & 0 & 1 \\
            0 & 1 & 0 & -1 \\
            1 & 0 & -1 & 0
        \end{pmatrix}
        \begin{pmatrix} 0 \\ 0 \\ 0 \\ 1 \end{pmatrix}
        = \inv{\sqrt{2}}
        \begin{pmatrix} 0 \\ 1 \\ -1 \\ 0 \end{pmatrix}
        \equiv \frac{\ket{01} - \ket{10}}{\sqrt{2}} = \ket{\Psi_-}.
    \]
    
    En la base de Bell $B_{Bell} = \{ \ket{\Phi_\pm}, \ket{\Psi_\pm} \}$, los operadores de medición serán proyectores $M^{(Bell)}_j$ diagonales. Por el contrario, los proyectores a estados en la base computacional $B_{Comp} = \{ \ket{00}, \ket{01}, \ket{10}, \ket{11} \}$ no serán diagonales expresados en la base de Bell. Si identificamos a los estados de Bell con vectores columna
    \[ \ket{\Phi_+} \equiv \begin{pmatrix} 1 \\ 0 \\ 0 \\ 0 \end{pmatrix}, \quad \ket{\Psi_+} \equiv \begin{pmatrix} 0 \\ 1 \\ 0 \\ 0 \end{pmatrix}, \quad \ket{\Phi_-} \equiv \begin{pmatrix} 0 \\ 0 \\ 1 \\ 0 \end{pmatrix}, \quad \ket{\Psi_-} \equiv \begin{pmatrix} 0 \\ 0 \\ 0 \\ 1 \end{pmatrix}, \]
    entonces por lo demostrado en la primera parte, $W$ constituye la matriz de cambio de base $B_{Bell} \to B_{Comp}$. Y, como los proyectores en ambas bases serán ortogonales, es natural entonces escribir:
    \[ [M^{B}_j]_C = W [M^{B}_j]_B W^\dagger, \]
    siendo $M^{B}_j$ el operador de proyección de estados de Bell, $[M^{B}_j]_B$ su representación matricial (diagonal) en la base de Bell, y $[M^{B}_j]_C$ su representación matricial (no diagonal) en la base de Bell.
    
    En términos de circuitos cuánticos, 
    \[
        \begin{array}{c}
            \Qcircuit @C=1.4em @R=1.2em {
                & \multigate{1}{W} & \qw \\
                & \ghost{W} & \qw \\
            }
        \end{array}
        \equiv
        \begin{array}{c}
            \Qcircuit @C=1.4em @R=1.2em {
                & \gate{H} & \ctrl{1} & \qw \\
                & \qw      & \targ & \qw \\
            }
        \end{array}
    \]
    
    \[
        \begin{array}{c}
            \Qcircuit @C=1.4em @R=1.2em {
                & \multigate{1}{W^\dagger} & \qw \\
                & \ghost{W^\dagger} & \qw \\
            }
        \end{array}
        \equiv \left(
        \begin{array}{c}
            \Qcircuit @C=1.4em @R=1.2em {
                & \gate{H} & \ctrl{1} & \qw \\
                & \qw      & \targ & \qw \\
            }
        \end{array}
        \right)^{-1} =
        \begin{array}{c}
            \Qcircuit @C=1.4em @R=1.2em {
                & \targ & \gate{H} & \qw \\
                & \ctrl{-1} & \qw  & \qw \\
            }
        \end{array}
    \]
    
    
    
    %-------------------------------------------------------------------------------------------------------
    %   Problema II.3
    %-------------------------------------------------------------------------------------------------------
    \item Sea $U_S = U_X \bar{U}_X U_X$, con $\bar{U}_X = \mathds{1} \otimes \dyad{0}{0} + X \otimes \dyad{1}{1}$. Matricialmente,
    \begin{align}
            \bar{U}_X = \mathds{1} \otimes \dyad{0}{0} + X \otimes \dyad{1}{1} &=
            \begin{pmatrix}
                \mqty{\pmat{0}}
            \end{pmatrix}
            \otimes
            \begin{pmatrix}
                1 & 0 \\
                0 & 0
            \end{pmatrix}
            +
            \begin{pmatrix}
                \mqty{\pmat{1}}
            \end{pmatrix}
            \otimes
            \begin{pmatrix}
                0 & 0 \\
                0 & 1
            \end{pmatrix} \\
            &=
            \begin{pmatrix}
                1 & 0 & 0 & 0 \\
                0 & 0 & 0 & 0 \\
                0 & 0 & 1 & 0 \\
                0 & 0 & 0 & 0
            \end{pmatrix}
            +
            \begin{pmatrix}
                0 & 0 & 0 & 0 \\
                0 & 0 & 0 & 1 \\
                0 & 0 & 0 & 0 \\
                0 & 1 & 0 & 0
            \end{pmatrix} \\
            &=
            \begin{pmatrix}
                1 & 0 & 0 & 0 \\
                0 & 0 & 0 & 1 \\
                0 & 0 & 1 & 0 \\
                0 & 1 & 0 & 0
            \end{pmatrix}.
        \end{align}
        
        Luego,
        \begin{align}
            U_S = U_X \bar{U}_X U_X = U_X
            \begin{pmatrix}
                1 & 0 & 0 & 0 \\
                0 & 0 & 0 & 1 \\
                0 & 0 & 1 & 0 \\
                0 & 1 & 0 & 0
            \end{pmatrix}
            \begin{pmatrix}
                \mqty{\mqty{\pmat{0}} & \mqty{\zmat{2}{2}} \\ \mqty{\zmat{2}{2}} & \mqty{\pmat{1}}}
            \end{pmatrix} &=
            \begin{pmatrix}
                \mqty{\mqty{\pmat{0}} & \mqty{\zmat{2}{2}} \\ \mqty{\zmat{2}{2}} & \mqty{\pmat{1}}}
            \end{pmatrix}
            \begin{pmatrix}
                1 & 0 & 0 & 0 \\
                0 & 0 & 1 & 0 \\
                0 & 0 & 0 & 1 \\
                0 & 1 & 0 & 0
            \end{pmatrix} \\
            &=
            \begin{pmatrix}
                1 & 0 & 0 & 0 \\
                0 & 0 & 1 & 0 \\
                0 & 1 & 0 & 0 \\
                0 & 0 & 0 & 1
            \end{pmatrix}.
        \end{align}
        
        Por lo tanto, $U_S \ket{00} = \ket{00}$, $U_S \ket{11} = \ket{11}$, $U_S \ket{01} = \ket{10}$ y $U_S \ket{10} = \ket{01}$. En general:
        \[ U_S \ket{ab} = \ket{ba}. \]
        
        El circuito correspondiente será:
        \[
            \begin{array}{c}
                \Qcircuit @C=1.4em @R=1.2em {
                    \lstick{\ket{a}} & \multigate{1}{U_S} & \rstick{\ket{b}} \qw \\
                    \lstick{\ket{b}} & \ghost{U_S} & \rstick{\ket{a}} \qw
                }
            \end{array}
            \quad \quad
            =
            \quad \quad
            \begin{array}{c}
                \Qcircuit @C=1.4em @R=1.2em {
                    \lstick{\ket{a}} & \ctrl{1}  & \targ      & \ctrl{1}  & \rstick{\ket{b}} \qw \\
                    \lstick{\ket{b}} & \targ     & \ctrl{-1}  & \targ     & \rstick{\ket{a}} \qw \\
                }
            \end{array}
            \quad .
        \]
        
\end{enumerate}

\section{Teleportación Cuántica}
\begin{enumerate}
    %-------------------------------------------------------------------------------------------------------
    %   Problema III.1
    %-------------------------------------------------------------------------------------------------------
    \item Tenemos un proceso de teleportación, dado por el circuito:
    \[
        \begin{array}{c}
            \Qcircuit @C=.7em @R=.4em @! {
                \lstick{\ket{\phi}_C} & \qw & \qw \barrier{2} & \ctrl{1} & \gate{H} \barrier{2} & \qw & \qw & \meter \\
                \lstick{\ket{0}_A} & \gate{H} & \ctrl{1} & \targ & \qw & \qw & \meter & \cwx \\
                \lstick{\ket{0}_B} & \qw & \targ & \qw & \qw & \qw & \gate{X} \cwx & \gate{Z} \cwx & \rstick{\ket{\phi}} \qw
            }
        \end{array}
    \]
    
    En la primera barrera, el sistema se encuentra en un estado establecido de Bell entre $A$ y $B$, y $C$ en un estado arbitrario $\ket{\phi} = \alpha \ket{0} + \beta \ket{1}$. Por lo tanto,
    \[ \ket{\Psi_{CAB}} = \left( \alpha \ket{0} + \beta \ket{1} \right) \left( \frac{\ket{00} + \ket{11}}{\sqrt{2}} \right). \]
    \begin{enumerate}
        \item En la segunda sección,
        \begin{align}
            \ket{\psi_{CAB}} &\xrightarrow{U_X} \alpha \ket{0} \left( \frac{\ket{00} + \ket{11}}{\sqrt{2}} \right) + \beta \ket{1} \left( \frac{\ket{10} + \ket{01}}{\sqrt{2}} \right) \\
                &\xrightarrow{H} \alpha \left( \frac{\ket{0} + \ket{1}}{\sqrt{2}} \right) \left( \frac{\ket{00} + \ket{11}}{\sqrt{2}} \right) + \beta \left( \frac{\ket{0} - \ket{1}}{\sqrt{2}} \right) \left( \frac{\ket{10} + \ket{01}}{\sqrt{2}} \right) \\
                &= \inv{2} \left\{ \ket{00} (\alpha \ket{0} + \beta \ket{1}) + \ket{01} (\alpha \ket{1} + \beta \ket{0}) + \ket{10} (\alpha \ket{0} - \beta \ket{1}) \right. \\
                &\quad \quad \left. + \ket{11} (\alpha \ket{1} - \beta \ket{0}) \right\}.
        \end{align}
        El último estado corresponderá al estado del sistema antes de que $A$ realice la medida. La matriz reducida de $B$ en ese instante entonces estará dada por:
        \begin{align}
            \rho_B &= \inv{4} \left\{ \left[ \alpha^2 \dyad{0}{0} + \beta^2 \dyad{1}{1} + \cancel{\alpha\beta (\dyad{0}{1} + \dyad{1}{0})} \right] \right. \\
            &\quad \quad \left. + \left[ \alpha^2 \dyad{1}{1} + \beta^2 \dyad{0}{0} - \cancel{\alpha\beta (\dyad{0}{1} + \dyad{1}{0})} \right] \right. \\
            &\quad \quad \left. + \left[ \alpha^2 \dyad{1}{1} + \beta^2 \dyad{0}{0} + \cancel{\alpha\beta (\dyad{0}{1} + \dyad{1}{0})} \right] \right. \\
            &\quad \quad \left. + \left[ \alpha^2 \dyad{0}{0} + \beta^2 \dyad{1}{1} - \cancel{\alpha\beta (\dyad{0}{1} + \dyad{1}{0})} \right] \right\} \\
            &= \inv{4} \left\{ 2(\underbrace{\alpha^2 + \beta^2}_{=1}) \dyad{0}{0} + 2(\underbrace{\alpha^2 + \beta^2}_{=1}) \dyad{1}{1} \right\} \\
            &= \inv{2} \dyad{0}{0} + \inv{2} \dyad{1}{1} = \frac{\mathds{1}}{2}.
        \end{align}
        Este resultado es completamente esperable, ya que en todo la segunda sección no se operó sobre el qubit $B$.
        
        \item Luego de que $A$ realice la medida (y se efectúe la correspondiente operación $X$ en $B$), tendremos:
        \begin{align}
            &\inv{2} \left\{ \ket{00} (\alpha \ket{0} + \beta \ket{1}) + \ket{01} (\alpha \ket{1} + \beta \ket{0}) + \ket{10} (\alpha \ket{0} - \beta \ket{1}) + \ket{11} (\alpha \ket{1} - \beta \ket{0}) \right\} \\
                &\xrightarrow[]{U_{X, A \to B}} \inv{2} \left\{ \ket{00} (\alpha \ket{0} + \beta \ket{1}) + \ket{10} (\alpha \ket{0} - \beta \ket{1}) + \ket{01} (\alpha \ket{0} + \beta \ket{1}) \right. \\
                &\quad \quad \quad \quad \quad \quad \quad \left. + \ket{11} (\alpha \ket{0} - \beta \ket{1}) \right\} \\
                &\quad \quad \quad \quad = \ket{\psi^{(A)}_{CAB}}.
        \end{align}
        Si conocemos el resultado de la medida, entonces:
        \begin{align}
            \ket{\psi_{A}} = \ket{0} &\implies \ket{\psi^{(A)}_{CAB}} = \frac{\sqrt{\mathcal{N}}}{2} \left\{ \ket{00} (\alpha \ket{0} + \beta \ket{1}) + \ket{10} (\alpha \ket{0} - \beta \ket{1}) \right\} \\
                &\implies \rho^{(A)}_B = \frac{\mathcal{N}}{4} \left[ (\alpha \ket{0} + \beta \ket{1}) (\alpha \bra{0} + \beta \bra{1}) + (\alpha \ket{0} - \beta \ket{1}) (\alpha \bra{0} - \beta \bra{1})\right] \\
                &\quad \quad \quad \quad =\frac{\mathcal{N}}{2} (\alpha^2 \dyad{0}{0} + \beta^2 \dyad{1}{1}) \\
                &\quad \quad \quad \quad =\frac{1}{\alpha^2 + \beta^2} (\alpha^2 \dyad{0}{0} + \beta^2 \dyad{1}{1})
        \end{align}
        y
        \begin{align}
            \ket{\psi_{A}} = \ket{1} &\implies \ket{\psi^{(A)}_{CAB}} = \frac{\sqrt{\mathcal{N}}}{2} \left\{ \ket{01} (\alpha \ket{0} + \beta \ket{1}) + \ket{11} (\alpha \ket{0} - \beta \ket{1}) \right\} \\
                &\implies \rho^{(A)}_B = \frac{\mathcal{N}}{4} \left[ (\alpha \ket{0} + \beta \ket{1}) (\alpha \bra{0} + \beta \bra{1}) + (\alpha \ket{0} - \beta \ket{1}) (\alpha \bra{0} - \beta \bra{1})\right] \\
                &\quad \quad \quad \quad =\frac{\mathcal{N}}{2} (\alpha^2 \dyad{0}{0} + \beta^2 \dyad{1}{1}) \\
                &\quad \quad \quad \quad =\frac{1}{\alpha^2 + \beta^2} (\alpha^2 \dyad{0}{0} + \beta^2 \dyad{1}{1}).
        \end{align}
        
        
        \item Si desconocemos el resultado de la medida, entonces el estado promedio reducido en $B$ será:
        \begin{align}
            \rho_B &= \inv{4} \left\{ \left[ \alpha^2 \dyad{0}{0} + \beta^2 \dyad{1}{1} + \cancel{\alpha\beta (\dyad{0}{1} + \dyad{1}{0})} \right] \right. \\
            &\quad \quad \left. + \left[ \alpha^2 \dyad{1}{1} + \beta^2 \dyad{0}{0} - \cancel{\alpha\beta (\dyad{0}{1} + \dyad{1}{0})} \right] \right. \\
            &\quad \quad \left. + \left[ \alpha^2 \dyad{1}{1} + \beta^2 \dyad{0}{0} + \cancel{\alpha\beta (\dyad{0}{1} + \dyad{1}{0})} \right] \right. \\
            &\quad \quad \left. + \left[ \alpha^2 \dyad{0}{0} + \beta^2 \dyad{1}{1} - \cancel{\alpha\beta (\dyad{0}{1} + \dyad{1}{0})} \right] \right\} \\
            &= \inv{4} \left\{ 2(\underbrace{\alpha^2 + \beta^2}_{=1}) \dyad{0}{0} + 2(\underbrace{\alpha^2 + \beta^2}_{=1}) \dyad{1}{1} \right\} \\
            &= \inv{2} \dyad{0}{0} + \inv{2} \dyad{1}{1} = \frac{\mathds{1}}{2}.
        \end{align}
        
    \end{enumerate}
    
    
    
    %-------------------------------------------------------------------------------------------------------
    %   Problema III.2
    %-------------------------------------------------------------------------------------------------------
    \item A partir de los desarrollos del punto II.1, luego de la aplicación de la última medida y compuerta $U_Z$, vemos que el estado final del sistema será
    \begin{align}
        &\inv{2} \left\{ \ket{00} (\alpha \ket{0} + \beta \ket{1}) + \ket{10} (\alpha \ket{0} - \beta \ket{1}) + \ket{01} (\alpha \ket{0} + \beta \ket{1}) + \ket{11} (\alpha \ket{0} - \beta \ket{1}) \right\} \\
            &\xrightarrow[]{U_{Z, C \to B}} \inv{2} \left\{ \ket{00} (\alpha \ket{0} + \beta \ket{1}) + \ket{10} (\alpha \ket{0} + \beta \ket{1}) + \ket{01} (\alpha \ket{0} + \beta \ket{1}) + \ket{11} (\alpha \ket{0} + \beta \ket{1}) \right\} \\
            &\quad \quad \quad \quad = \inv{2} \left( \ket{00} + \ket{01} + \ket{10} + \ket{11} \right) (\alpha \ket{0} + \beta \ket{1}) \\
            &\quad \quad \quad \quad = \inv{2} \left( \ket{00} + \ket{01} + \ket{10} + \ket{11} \right) \ket{\phi} \\
            &\quad \quad \quad \quad = \ket{\psi^{(Final)}_{CAB}}.
    \end{align}
    La matriz densidad reducida $\rho_B$ del en el estado final estará dada entonces trivialmente por
    \[ \rho^{(F)}_B = \dyad{\phi}{\phi}. \]
    
    \begin{enumerate}
        \item Sea un estado no puro general $\rho_C = p_0 \dyad{\psi_0}{\psi_0} + (1 - p_0) \dyad{\psi_1}{\psi_1}$. Como el proceso de teleportación está compuesto de operaciones lineales en cada uno de sus pasos, podemos aplicar el procedimiento a cada estado $\psi_0$ y $\phi_1$ independientemente, conformados por las matrices densidad $\rho^{0} = \dyad{\psi_0}{\psi_0}$ y $\rho^{1} = \dyad{\psi_1}{\psi_1}$:
        \[ T\left\{ \dyad{\psi_0}{\psi_0}_C \otimes \dyad{\Phi_+}{\Phi_+}_{AB} \right\} = \frac{\mathds{1}_{CA}}{2} \otimes \dyad{\psi_0}{\psi_0}_{B}, \]
        \[ T\left\{ \dyad{\psi_1}{\psi_1}_C \otimes \dyad{\Phi_+}{\Phi_+}_{AB} \right\} = \frac{\mathds{1}_{CA}}{2} \otimes \dyad{\psi_1}{\psi_1}_{B}. \]
        
        Luego, por ser operaciones lineales, la teleportación del estado $\rho_C = p_0 \rho^{(0)}_C + (1 - p_0) \rho^{(1)}_C$ resultará en
        \begin{align}
            &T\left\{ p_0 \dyad{\psi_0}{\psi_0}_C \otimes \dyad{\Phi_+}{\Phi_+}_{AB} + (1 - p_0) \dyad{\psi_1}{\psi_1}_C \otimes \dyad{\Phi_+}{\Phi_+}_{AB} \right\} \\
                &\quad \quad = p_0 T\left\{ \dyad{\psi_0}{\psi_0}_C \otimes \dyad{\Phi_+}{\Phi_+}_{AB} \right\} + (1 - p_0) T\left\{ \dyad{\psi_1}{\psi_1}_C \otimes \dyad{\Phi_+}{\Phi_+}_{AB} \right\} \\
                &\quad \quad = p_0 \frac{\mathds{1}_{CA}}{2} \otimes \dyad{\psi_0}{\psi_0}_{B} + (1 - p_0) \frac{\mathds{1}_{CA}}{2} \otimes \dyad{\psi_1}{\psi_1}_{B} \\
                &\quad \quad = \frac{\mathds{1}_{CA}}{2} \otimes \left[ p_0 \dyad{\psi_0}{\psi_0}_{B} + (1 - p_0) \dyad{\psi_1}{\psi_1}_{B} \right],
        \end{align}
        por lo que la matriz densidad reducida de $B$ será
        \[ \rho^{(F)}_B = p_0 \rho^{(0)} + (1 - p_0) \rho^{(1)}. \]
        
        
        \item Supongamos ahora que $C$ está inicialmente entrelazado con un cuarto sistema $D$:
        \[ \ket{\psi_{DC}} = \sqrt{p_0} \ket{0}_D \ket{\psi_0}_C + \sqrt{1 - p_0} \ket{1}_D \ket{\psi_1}_C. \]
        Como podemos apreciar en el circuito correspondiente, las operaciones de teleportación de $C$ a $B$ se dan de manera independiente del qubit $D$:
        \[
            \begin{array}{c}
                \Qcircuit @C=.7em @R=.4em @! {
                    \lstick{D} & \qw & \qw & \qw & \qw & \qw \\
                    \lstick{C} & \ctrl{1} & \gate{H} & \qw & \meter \inputgrouph{1}{2}{.75em}{\ket{\psi_{DC}}}{3em} \\
                    \lstick{A} & \targ & \qw & \meter & \cwx \\
                    \lstick{B} & \qw & \qw & \gate{X} \cwx & \gate{Z} \cwx \qw & \qw \inputgrouph{3}{4}{.75em}{\ket{\psi_{AB}}}{3em}
                }
            \end{array}
        \]
        Por lo tanto, la operación de teleportación $T_2$ puede definirse en términos del proceso de teleportación ya definido en III.1, obteniendo:
        \begin{align}
            T_2 \ket{\psi_{DCAB}} &= \sqrt{p_0} \ket{0}_D \otimes T\left\{ \ket{\psi_0}_C \otimes \ket{\Phi_+}_{AB} \right\} + \sqrt{1 - p_0} \ket{1}_D \otimes T\left\{ \ket{\psi_1}_C \otimes \ket{\Phi_+}_{AB} \right\} \\
            &= \sqrt{p_0} \ket{0}_D \otimes \ket{\Omega}_{CA} \otimes \ket{\psi_0}_D + \sqrt{1 - p_0} \ket{1}_D \otimes \ket{\Omega}_{CA} \otimes \ket{\psi_1}_D \\
            &= \ket{\Omega}_{CA} \otimes \left( \sqrt{p_0} \ket{0}_D \otimes \ket{\psi_0}_D + \sqrt{1 - p_0} \ket{1}_D \otimes \ket{\psi_1}_D \right),
        \end{align}
        donde $\ket{\Omega} = (\ket{00} + \ket{01} + \ket{10} + \ket{11})/2$. Luego,
        \[ \ket{\psi_{DB}} = \sqrt{p_0} \ket{0}_D \otimes \ket{\psi_0}_D + \sqrt{1 - p_0} \ket{1}_D \otimes \ket{\psi_1}_D. \]
        Podemos concluir que el proceso de teleportación ``transfiere'' el entrelazamiento con $D$ de $C$ a $B$.
        
        
        \item Supondremos un estado puro $\ket{\psi}_C = \alpha \ket{0} + \beta \ket{1}$ en el qubit a teleportar, utilizando ahora un estado de Bell $\ket{\Psi_-}_{AB} = \frac{\ket{01} - \ket{10}}{\sqrt{2}}$ en los qubits $A$ y $B$:
        \[ \ket{\psi_{CAB}} = (\alpha \ket{0}_C + \beta \ket{1}_C) \otimes \frac{\ket{01}_{AB} - \ket{10}_{AB}}{\sqrt{2}}. \]
        Luego de la aplicación del operador $U_{CNOT}$ entre los qubits $C$ y $A$, y el operador $H$ en $C$, obtendremos el estado
        \begin{align}
            &(\alpha \ket{0}_C + \beta \ket{1}_C) \otimes \frac{\ket{01}_{AB} - \ket{10}_{AB}}{\sqrt{2}} \\
            &\quad \xrightarrow{U_{X, CA}} \alpha \ket{0}_C \otimes \left( \frac{\ket{01}_{AB} - \ket{10}_{AB}}{\sqrt{2}} \right) + \beta \ket{1}_C \otimes \left( \frac{\ket{11}_{AB} - \ket{00}_{AB}}{\sqrt{2}} \right) \\
            &\quad \xrightarrow{H_{C}} \alpha \left(\frac{\ket{0}_C + \ket{1}_C}{\sqrt{2}} \right) \otimes \left( \frac{\ket{01}_{AB} - \ket{10}_{AB}}{\sqrt{2}} \right) + \beta \left(\frac{\ket{0}_C - \ket{1}_C}{\sqrt{2}} \right) \otimes \left( \frac{\ket{11}_{AB} - \ket{00}_{AB}}{\sqrt{2}} \right) \\
            &\quad = \inv{2} \left\{ \ket{00}_{CA} \otimes (\alpha \ket{1} - \beta \ket{0}) + \ket{11}_{CA} \otimes (-\alpha \ket{0} - \beta \ket{1}) + \ket{01}_{CA} \otimes (-\alpha \ket{0} + \beta \ket{1}) \right. \\
            &\quad \quad \quad \left. + \ket{10}_{CA} \otimes (\alpha \ket{1} + \beta \ket{0}) \right\} \\
            &\quad = \inv{2} \left\{ -\ket{00}_{CA} \otimes ZX \ket{\psi}_B - \ket{11}_{CA} \otimes \ket{\psi}_B - \ket{01}_{CA} \otimes Z \ket{\psi}_B + \ket{10}_{CA} \otimes X \ket{\psi}_B \right\}.
        \end{align}
        Luego, de haber obtenido el estado $\ket{00}_{CA}$, se deberán aplicar los operadores $XZ$ al qubit $B$. Así mismo, si se obtuvieron $\ket{01}_{CA}$ o $\ket{10}_{CA}$, se deberán aplicar en $B$ las operaciones $Z$ o $X$ respectivamente. En el caso de haber obtenido $\ket{11}_{CA}$, no se deberán realizar cambios al sistema.
        
    \end{enumerate}
    
    
\end{enumerate}
%\nocite{*}
%\printbibliography
\end{document}