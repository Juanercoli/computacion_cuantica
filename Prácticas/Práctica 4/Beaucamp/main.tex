\documentclass{scrartcl}
\usepackage[a4paper,margin=2cm,footskip=1cm]{geometry}
\setkomafont{disposition}{\normalfont\bfseries}

%\documentclass{article}

\usepackage[table,xcdraw]{xcolor}
\usepackage{tikz}
\usetikzlibrary{angles,quotes}
\usetikzlibrary{babel}

\usepackage{booktabs}
\usepackage[utf8]{inputenc}
\usepackage[spanish, es-nodecimaldot]{babel}
\usepackage[per-mode=symbol]{siunitx}
\usepackage{graphicx}
\usepackage{subcaption}
\usepackage{caption}
\usepackage{mathtools}
\usepackage{amsmath}
\usepackage{slashed}
\usepackage{dsfont}
\usepackage{float}
\usepackage{multicol}
\usepackage{wrapfig}
\usepackage{lipsum}
\usepackage{textcomp}
\usepackage{gensymb}
\usepackage{longtable}
\usepackage{supertabular}
\usepackage{hhline}
\usepackage{enumerate}
\usepackage{multirow}
\usepackage{amssymb}
\usepackage{tabularx}
\usepackage{ragged2e}
\usepackage{rotating}
\usepackage{cancel}
\usepackage{physics}
\usepackage[framemethod=default]{mdframed}
\usepackage{csquotes}
%\usepackage[backend=biber, style=numeric, sorting=none]{biblatex}
\usepackage{qcircuit}
\usepackage{bm}

\renewcommand{\figurename}{Figura}
\renewcommand{\spanishtablename}{Tabla}
\newcommand{\inv}[1]{\frac{1}{#1}}
\newcommand{\uv}[1]{\hat{\mathbf{#1}}}
\newcommand{\uvs}[1]{\, \uv{#1}}

\newcommand{\realSet}{\mathbb{R}}
\newcommand{\complexSet}{\mathbb{C}}
\newcommand{\oref}{$\mathcal{O}$ }
\newcommand{\opref}{$\mathcal{O}'$ }
\newcommand{\oppref}{$\mathcal{O}''$ }

\def\residue{\mathop{\text{Res}}}

\setlength{\tabcolsep}{19pt}

\DeclareSIUnit\clight{\text{\ensuremath{c}}}
\DeclareSIUnit\MeV{\mega\electronvolt}
\DeclareSIUnit\GeV{\giga\electronvolt}
\DeclareSIUnit\MeVpc{\MeV\per\clight\squared}
\DeclareSIUnit\GeVpc{\GeV\per\clight\squared}

\newcommand{\sinc}{\text{sinc}}
\newcommand{\E}{\vb{E}}
\newcommand{\B}{\vb{B}}
\newcommand{\x}{\vb{x}}
\newcommand{\y}{\vb{y}}
\newcommand{\z}{\vb{z}}
\newcommand{\p}{\vb{p}}
\renewcommand{\k}{\vb{k}}
\newcommand{\Lag}{\mathcal{L}}
\newcommand{\Ham}{\mathcal{H}}

\newcommand{\tx}{\tilde{x}}

\renewcommand{\vb}[1]{\bm{#1}}

\renewcommand{\a}[1]{\hat{a}_{#1}}
\newcommand{\ad}[1]{\hat{a}_{#1}^\dagger}

\DeclareRobustCommand{\[}{\begin{equation}}
\DeclareRobustCommand{\]}{\end{equation}}
\mathtoolsset{showonlyrefs}

\allowdisplaybreaks

%\bibliography{bibliography}

%----------------------------------------------------------------------------------------
%	DOCUMENT INFORMATION
%----------------------------------------------------------------------------------------

\title{Teoría de la Información Cuántica}
\subtitle{Práctica 4 - Año 2020}
\author{\textsc{Beaucamp}, Jean Yves}
\date{}

\begin{document}

\maketitle

\section{Evolución de sistemas cuánticos abiertos}
\begin{enumerate}
    
    %-------------------------------------------------------------------------------------------------------
    %   Problema I.1
    %-------------------------------------------------------------------------------------------------------
    \item Dado un estado inicial producto $\rho_{AB} = \rho_a \otimes \rho_B$ de un sistema bipartito, y un operador evolución $U_{AB} = \exp{-iH_{AB} t}$, el estado reducido final $\rho_A'$ estará dado por
    \[ \rho_A' = \Tr_B U_{AB} \rho_A \otimes \rho_B U_{AB}^\dagger = \sum_j \bra{j}_B U_{AB} \rho_A \otimes \rho_B U_{AB}^\dagger \ket{j}_B. \]
    Suponiendo un estado mezcla general para $B$, dado por $\rho_B = \sum_l \dyad{l}{l}_B P_l$, entonces
    \begin{align}
        \rho_A' &= \sum_j \bra{j}_B U_{AB} \ \rho_A \otimes \rho_B \ U_{AB}^\dagger \ket{j}_B \\
        &= \sum_{j, k} \underbrace{\bra{j}_B U_{AB} \ket{l}_B \sqrt{P_l}}_{K_{jl}} \rho_A \underbrace{\sqrt{P_l} \bra{l}_B U_{AB}^\dagger \ket{j}_B}_{K_{jl}^\dagger} \\
        &= \sum_{j, k} K_{jl} \rho_A K_{jl}^\dagger.
    \end{align}
    Luego, los operadores de Kraus
    \[ K_{jl} = \bra{j}_B U_{AB} \ket{l}_B \sqrt{P_l} \label{eq:I_1_1} \]
    darán la evolución del subsistema $A$. Además,
    \begin{align}
        \sum_{j,l} K_{jl}^\dagger K_{jl} = \sum_{j, l} \sqrt{P_l} \bra{l}_B \rho_B U_{AB}^\dagger \ket{j}_B \bra{j}_B U_{AB} \ket{l}_B \sqrt{P_l} &= \sum_{l} P_l \bra{l}_B U_{AB}^\dagger U_{AB} \ket{l}_B \\
            &= \sum_{l} P_l \ip{l}{l}_B \mathds{1}_A \\
            &= \sum_{l} P_l \mathds{1}_A \\
            &= \mathds{1}_A.
    \end{align}
    
    En el caso particular en que se trata de un estado inicial puro $\rho_B = \dyad{\psi}{\psi}$, entonces los operadores de Kraus tomarán la forma explícita
    \[ E_\alpha = \bra{\alpha}_B U_{AB} \ket{\psi}_B. \]
    
    
    
    %-------------------------------------------------------------------------------------------------------
    %   Problema I.2
    %-------------------------------------------------------------------------------------------------------
    \item
    \begin{enumerate}[a)]
        \item Para el caso de una evolución por un operador Control-NOT, dada por $U_{AB} = \dyad{0}{0} \otimes \mathds{1} + \dyad{1}{1} \otimes X$, aplicada a un estado producto de dos qubits $\rho_{AB} = \rho_A \otimes \dyad{\psi}{\psi}_B$, el estado parcial final de el subsistema $A$ será
        \begin{align}
            \rho_{AB}' = U_{AB} \ \rho_{AB} \ U_{AB}^\dagger &= \dyad{0}{0} \rho_A \dyad{0}{0} \otimes \dyad{\psi}{\psi}_B + \dyad{0}{0} \rho_A \dyad{1}{1} \otimes X \dyad{\psi}{\psi}_B \\
                &\quad + \dyad{1}{1} \rho_A \dyad{0}{0} \otimes X \dyad{\psi}{\psi}_B + \dyad{1}{1} \rho_A \dyad{1}{1} \otimes X^2 \dyad{\psi}{\psi}_B \\
                &= \dyad{0}{0} \rho_A \dyad{0}{0} \otimes \dyad{\psi}{\psi}_B \\
                &\quad + \dyad{0}{0} \rho_A \dyad{1}{1} \otimes X \dyad{\psi}{\psi}_B \\
                &\quad + \dyad{1}{1} \rho_A \dyad{0}{0} \otimes X \dyad{\psi}{\psi}_B \\
                &\quad + \dyad{1}{1} \rho_A \dyad{1}{1} \otimes \dyad{\psi}{\psi}_B.
        \end{align}
        
        Si $\ket{\psi}_B = \alpha \ket{+} + \beta \ket{-}$, entonces $X \ket{\psi}_B = \alpha \ket{+} - \beta \ket{-}$. Luego,
        \begin{align}
            \rho_A' = \Tr_B \rho_{AB}' &= \dyad{0}{0} \rho_A \dyad{0}{0} \qty(\abs{\alpha}^2 + \abs{\beta}^2) \\
                &\quad + \dyad{0}{0} \rho_A \dyad{1}{1} \qty(\abs{\alpha}^2 - \abs{\beta}^2) \\
                &\quad + \dyad{1}{1} \rho_A \dyad{0}{0} \qty(\abs{\alpha}^2 - \abs{\beta}^2) \\
                &\quad + \dyad{1}{1} \rho_A \dyad{1}{1} \qty(\abs{\alpha}^2 + \abs{\beta}^2) \\
                &= \abs{\alpha^2} \rho_A + \abs{\beta}^2 Z \rho_A Z,
        \end{align}
        utilizando que
        \[
            Z \rho_A Z = \begin{pmatrix} 1 & 0 \\ 0 & -1 \end{pmatrix} \begin{pmatrix} \rho^A_{00} & \rho^A_{01} \\ \rho^A_{10} & \rho^A_{11} \end{pmatrix} \begin{pmatrix} 1 & 0 \\ 0 & -1 \end{pmatrix} = \begin{pmatrix} 1 & 0 \\ 0 & -1 \end{pmatrix} \begin{pmatrix} \rho^A_{00} & -\rho^A_{01} \\ \rho^A_{10} & -\rho^A_{11} \end{pmatrix} = \begin{pmatrix} \rho^A_{00} & -\rho^A_{01} \\ -\rho^A_{10} & \rho^A_{11} \end{pmatrix}.
        \]
        
        Escribiendo el estado final como
        \[ \rho_A' = (1 - p) \rho_A + p Z \rho_A Z =
            \begin{pmatrix}
                \rho^A_{00} & (1 - 2p) \ \rho^A_{01} \\
                (1 - 2p) \ \rho^A_{10} & \rho^A_{11}
            \end{pmatrix}, \label{eq:I_2_a_1}
        \]
        podemos identificar $p = \abs{\beta}^2$ y $1 - p = \abs{\alpha}^2$, consistente con la condición de normalización de $\ket{\psi}_B$. Además, los operadores de Kraus asociados a la evolución no-unitaria serán
        \[ E_0 = \sqrt{1 - p} \mathds{1}_A = \abs{\alpha} \mathds{1}_A, \quad E_1 = \sqrt{p} Z = \abs{\beta} Z. \]
        
        Observando la representación matricial del estado $\rho_A'$ exhibida en \eqref{eq:I_2_a_1}, interpretamos la evolución del sistema como la disminución de los elementos no-diagonales. Para el estado inicial $\ket{\psi}_B = \frac{\ket{+} + \ket{-}}{\sqrt{2}} = \ket{0}$, $p = 1/2$ y el sistema presentará decoherencia completa. Así mismo, para $\ket{\psi}_B = \ket{+}$, $p = 0$ y el subsistema $A$ no evolucionará, siendo $\rho_A' = \rho_A$. La entropía asociada a $\rho_A$ tenderá solo a aumentar, al estar eliminando información del sistema conforme $p \to 1/2$.
        
        
        \item Para un estado inicial producto general $\rho_{AB} = \rho_A \otimes \rho_B$, con $\rho_B = \sum_l P_l \dyad{l}{l}_B$, la evolución del subsistema $A$ estará determinada por los operadores de Kraus \eqref{eq:I_1_1}:
        \[ K_{jl} = \mel{j}{U_{AB}}{l}_B \sqrt{P_l} = \dyad{0}{0}_A \ip{j}{l}_B \sqrt{P_l} + \dyad{1}{1}_A \mel{j}{X}{l}_B \sqrt{P_l}. \]
        Suponiendo un sistema de dos qubits, y utilizando la base $\{ \ket{\pm} \}$ para $\ket{j}$, 
        \begin{align}
            K_{\pm l} = \mel{\pm}{U_{AB}}{l}_B \sqrt{P_l} &= \dyad{0}{0}_A \ip{\pm}{l}_B \sqrt{P_l} + \dyad{1}{1}_A \mel{\pm}{X}{l}_B \sqrt{P_l} \\
                &= \dyad{0}{0}_A \ip{\pm}{l}_B \sqrt{P_l} \pm \dyad{1}{1}_A \ip{\pm}{l}_B \sqrt{P_l} \\
                &= (\dyad{0}{0}_A \pm \dyad{1}{1}) \ip{\pm}{l}_B \sqrt{P_l}.
        \end{align}
        
        Si $\rho_B = \mathds{1}/2$ en la base computacional, entonces
        \begin{align}
            K_{jl} = \mel{j}{U_{AB}}{l}_B \inv{\sqrt{2}} &= \dyad{0}{0}_A \ip{j}{l}_B \inv{\sqrt{2}} + \dyad{1}{1}_A \mel{j}{X}{l}_B \inv{\sqrt{2}} \\
                &= \dyad{0}{0}_A \delta_{jl} \inv{\sqrt{2}} + \dyad{1}{1}_A (\delta_{j0} \delta_{l1} + \delta_{j1} \delta_{l0}) \inv{\sqrt{2}} \\
        \end{align}
        \[ \therefore K_{00} = K_{11} = \inv{\sqrt{2}} \dyad{0}{0}_A, \quad \quad K_{01} = K_{10} = \inv{\sqrt{2}} \dyad{1}{1}_A. \]
        
\end{enumerate}
        
    %-------------------------------------------------------------------------------------------------------
    %   Problema I.3
    %-------------------------------------------------------------------------------------------------------
    \item Sean los operadores de Kraus $E_1 = \dyad{0}{0} + \sqrt{1 - p} \dyad{1}{1}$ y $E_2 = \sqrt{p} \dyad{0}{1}$, con $p \in [0, 1]$.
    \begin{enumerate}[a)]
        \item Los operadores $\{ E_1, E_2 \}$ satisfacen
        \begin{align}
            \sum_{i = 1, 2} E_i^\dagger E_i &= E_1^\dagger E_1 + E_2^\dagger E_2 \\
                &= \ket{0} \ip{0}{0} \bra{0} + (1 - p) \ket{1} \ip{1}{1} \bra{1} + p \ket{1} \ip{0}{0} \bra{1} \\
                &= \dyad{0}{0} + (1 - \cancel{p}) \dyad{1}{1} + \cancel{p} \dyad{1}{1} \\
                &= \dyad{0}{0} + \dyad{1}{1} \\
                &= \mathds{1}
        \end{align}
        para todo $p \in [0, 1]$, pero no se trata de operadores unitales, ya que
        \begin{align}
            \sum_{i = 1, 2} E_i E_i^\dagger &= E_1 E_1^\dagger + E_2 E_2^\dagger \\
                &= \ket{0} \ip{0}{0} \bra{0} + (1 - p) \ket{1} \ip{1}{1} \bra{1} + p \ket{0} \ip{1}{1} \bra{0} \\
                &= \dyad{0}{0} + (1 - p) \dyad{1}{1} + p \dyad{0}{0} \\
                &= (1 + p) \dyad{0}{0} + (1 - p) \dyad{1}{1},
        \end{align}
        siendo iguales a la identidad $\mathds{1}$ si y solo si $p = 0$.
        
        
        \item Como se ha demostrado en la práctica 1, un operador densidad $\rho_A$ general de 1 qubit puede ser representado en términos de las matrices de Pauli como
        \[ \rho_A = \inv{2} (\mathds{1} + \vb{r} \vdot \vb{\sigma}). \]
        Luego, la evolución del subsistema cuántico $\rho_A$ dada por los operadores de Kraus $E_1$ y $E_2$ antes definidos resultará en
        \[ \rho_A' = E_1 \rho_A E_1^\dagger + E_2 \rho_A E_2^\dagger = E_1 \rho_A E_1 + E_2 \rho_A E_2^\dagger. \]
        A continuación, evaluaremos individualmente los elementos de la sumatoria:
        \begin{align}
            E_1 \rho_A E_1 &= \inv{2} \left( E_1 E_1 + r_x \ E_1 \sigma_x E_1 + r_y \ E_1 \sigma_y E_1 + r_z \ E_1 \sigma_z E_1 \right) \\
                &= \inv{2} \left( \begin{pmatrix} 1 & 0 \\ 0 & \sqrt{1 - p} \end{pmatrix} \begin{pmatrix} 1 & 0 \\ 0 & \sqrt{1 - p} \end{pmatrix} \right. \\
                &\quad \quad \left. + r_x \ \begin{pmatrix} 1 & 0 \\ 0 & \sqrt{1 - p} \end{pmatrix} \begin{pmatrix} 0 & 1 \\ 1 & 0 \end{pmatrix} \begin{pmatrix} 1 & 0 \\ 0 & \sqrt{1 - p} \end{pmatrix} \right. \\
                &\quad \quad \left. + r_y \ \begin{pmatrix} 1 & 0 \\ 0 & \sqrt{1 - p} \end{pmatrix} \begin{pmatrix} 0 & -i \\ i & 0 \end{pmatrix} \begin{pmatrix} 1 & 0 \\ 0 & \sqrt{1 - p} \end{pmatrix} \right. \\
                &\quad \quad \left. + r_z \ \begin{pmatrix} 1 & 0 \\ 0 & \sqrt{1 - p} \end{pmatrix} \begin{pmatrix} 1 & 0 \\ 0 & -1 \end{pmatrix} \begin{pmatrix} 1 & 0 \\ 0 & \sqrt{1 - p} \end{pmatrix} \right) \\
                &= \inv{2} \left( \begin{pmatrix} 1 & 0 \\ 0 & 1 - p \end{pmatrix} \right. \\
                &\quad \quad \left. + r_x \ \begin{pmatrix} 1 & 0 \\ 0 & \sqrt{1 - p} \end{pmatrix} \begin{pmatrix} 0 & \sqrt{1 - p} \\ 1 & 0 \end{pmatrix} \right. \\
                &\quad \quad \left. + i r_y \ \begin{pmatrix} 1 & 0 \\ 0 & \sqrt{1 - p} \end{pmatrix} \begin{pmatrix} 0 & -\sqrt{1 - p} \\ 1 & 0 \end{pmatrix} \right. \\
                &\quad \quad \left. + r_z \ \begin{pmatrix} 1 & 0 \\ 0 & \sqrt{1 - p} \end{pmatrix} \begin{pmatrix} 1 & 0 \\ 0 & -\sqrt{1 - p} \end{pmatrix} \right) \\
                &= \inv{2} \left( \begin{pmatrix} 1 & 0 \\ 0 & 1 - p \end{pmatrix} \right. \\
                &\quad \quad \left. + r_x \ \begin{pmatrix} 0 & \sqrt{1 - p} \\ \sqrt{1 - p} & 0 \end{pmatrix}  \right. \\
                &\quad \quad \left. + i r_y \ \begin{pmatrix} 0 & -\sqrt{1 - p} \\ \sqrt{1 - p} & 0 \end{pmatrix} \right. \\
                &\quad \quad \left. + r_z \ \begin{pmatrix} 1 & 0 \\ 0 & -(1 - p) \end{pmatrix} \right) \\
                &= \inv{2} \begin{pmatrix} 1 & (r_x - i r_y) \sqrt{1 - p} \\ (r_x + i r_y) \sqrt{1 - p} & (1 - r_z)(1 - p) \end{pmatrix},
        \end{align}
        \begin{align}
            E_2 \rho_A E_2 &= \inv{2} \left( E_2 E_2^\dagger + r_x \ E_2 \sigma_x E_2^\dagger + r_y \ E_2 \sigma_y E_2^\dagger + r_z \ E_2 \sigma_z E_2^\dagger \right) \\
                &= \inv{2} \left( \begin{pmatrix} 0 & \sqrt{p} \\ 0 & 0 \end{pmatrix} \begin{pmatrix} 0 & 0 \\ \sqrt{p} & 0 \end{pmatrix} \right. \\
                &\quad \quad \left. + r_x \ \begin{pmatrix} 0 & \sqrt{p} \\ 0 & 0 \end{pmatrix} \begin{pmatrix} 0 & 1 \\ 1 & 0 \end{pmatrix} \begin{pmatrix} 0 & 0 \\ \sqrt{p} & 0 \end{pmatrix} \right. \\
                &\quad \quad \left. + r_y \ \begin{pmatrix} 0 & \sqrt{p} \\ 0 & 0 \end{pmatrix} \begin{pmatrix} 0 & -i \\ i & 0 \end{pmatrix} \begin{pmatrix} 0 & 0 \\ \sqrt{p} & 0 \end{pmatrix} \right. \\
                &\quad \quad \left. + r_z \ \begin{pmatrix} 0 & \sqrt{p} \\ 0 & 0 \end{pmatrix} \begin{pmatrix} 1 & 0 \\ 0 & -1 \end{pmatrix} \begin{pmatrix} 0 & 0 \\ \sqrt{p} & 0 \end{pmatrix} \right) \\
                &= \inv{2} \left( \begin{pmatrix} p & 0 \\ 0 & 0 \end{pmatrix} \right. \\
                &\quad \quad \left. + r_x \ \begin{pmatrix} 0 & \sqrt{p} \\ 0 & 0 \end{pmatrix} \begin{pmatrix} \sqrt{p} & 0 \\ 0 & 0 \end{pmatrix} \right. \\
                &\quad \quad \left. + i r_y \ \begin{pmatrix} 0 & \sqrt{p} \\ 0 & 0 \end{pmatrix} \begin{pmatrix} \sqrt{p} & 0 \\ 0 & 0 \end{pmatrix} \right. \\
                &\quad \quad \left. + r_z \ \begin{pmatrix} 0 & \sqrt{p} \\ 0 & 0 \end{pmatrix} \begin{pmatrix} 0 & 0 \\ -\sqrt{p} & 0 \end{pmatrix} \right) \\
                &= \inv{2} \left( \begin{pmatrix} p & 0 \\ 0 & 0 \end{pmatrix} + r_z \ \begin{pmatrix} -p & 0 \\ 0 & 0 \end{pmatrix} \right) \\
                &= \inv{2} \begin{pmatrix} (1 - r_z) p & 0 \\ 0 & 0 \end{pmatrix},
        \end{align}
        \begin{align}
            \implies \rho_A' &= \inv{2} \begin{pmatrix} 1 & (r_x - i r_y) \sqrt{1 - p} \\ (r_x + i r_y) \sqrt{1 - p} & (1 - r_z)(1 - p) \end{pmatrix} + \inv{2} \begin{pmatrix} (1 - r_z) p & 0 \\ 0 & 0 \end{pmatrix} \\
                &= \inv{2} \begin{pmatrix} 1 + (1 - r_z) p & (r_x - i r_y) \sqrt{1 - p} \\ (r_x + i r_y) \sqrt{1 - p} & (1 - r_z)(1 - p) \end{pmatrix}.
        \end{align}
        
    \end{enumerate}
    
    
    
    %-------------------------------------------------------------------------------------------------------
    %   Problema I.3
    %-------------------------------------------------------------------------------------------------------
    \item Consideremos un átomo de 2 niveles, con estados $\ket{0}$ y $\ket{1}$ de energías $0$ y $\varepsilon = \hbar \omega$ respectivamente, y estados de campo $\ket{0}$ t $\ket{1} = a_{\omega}^\dagger \ket{0}$. Supondremos una evolución unitaria del sistema $\rho_{AC} = \rho_{\text{Átomo}} \otimes \rho_{\text{Campo}}$ dada por
    \[
        U_{AC} =
        \begin{pmatrix}
            1 & 0 & 0 & 0 \\
            0 & \cos\theta & \sin\theta & 0 \\
            0 & -\sin\theta & \cos\theta & 0 \\
            0 & 0 & 0 & 1 \\
        \end{pmatrix}
    \]
    en la base estándar. Supondremos que el campo se encuentra inicialmente en el estado $\rho_C = \dyad{0}{0}$. Luego, el estado evolucionado del sistema Átomo-Campo estará dado por
    \begin{align}
        \rho_{AC}' = U_{AC} \ \rho_{AC} \ U_{AC}^\dagger &= U_{AC}
        \begin{pmatrix}
            \rho^A_{00} & \rho^A_{01} \\
            \rho^A_{10} & \rho^A_{11} \\
        \end{pmatrix}
        \otimes
        \begin{pmatrix}
            1 & 0 \\
            0 & 0 \\
        \end{pmatrix}
        U_{AC}^\dagger
        = U_{AC}
        \begin{pmatrix}
            \rho^A_{00} & 0 & \rho^A_{01} & 0 \\
            0 & 0 & 0 0 \\
            \rho^A_{10} & 0 & \rho^A_{11} & 0 \\
            0 & 0 & 0 & 0 \\
        \end{pmatrix}
        U_{AC}^\dagger \\
        &=
        \begin{pmatrix}
            \rho^A_{00} & \rho^A_{01} \sin\theta & \rho^A_{01} \cos\theta & 0 \\
            \rho^A_{10} \sin\theta & \rho^A_{11} \sin^2\theta & \rho^A_{11} \cos\theta \sin\theta & 0 \\
            \rho^A_{10} \cos\theta & \rho^A_{11} \cos\theta \sin\theta & \rho^A_{11} \cos^2\theta & 0 \\
            0 & 0 & 0 & 0 \\
        \end{pmatrix}
    \end{align}
    donde se ha evaluado el último producto en Mathematica.
    
    El estado parcial evolucionado del Átomo se encuentra determinado por la traza parcial de $\rho_{AC}'$:
    \[
        \rho_A' = \Tr_B \rho_{AC}' = 
        \begin{pmatrix}
            \rho^A_{00} + \rho^A_{11} \sin^2\theta & \rho^A_{01} \cos\theta \\
            \rho^A_{10} \cos\theta & \rho^A_{11} \cos^2\theta \\
        \end{pmatrix}.
        \label{eq:I_4_1}
    \]
    
    Escribiendo el operador de evolución como
    \begin{align}
        U_{AC} &= \dyad{0}{0} \otimes \dyad{0}{0} + \dyad{1}{1} \otimes \dyad{1}{1} \\
            &\quad + \cos\theta (\dyad{0}{0} \otimes \dyad{1}{1} + \dyad{1}{1} \otimes \dyad{0}{0}) \\
            &\quad + \sin\theta (\dyad{0}{1} \otimes \dyad{1}{0} - \dyad{1}{0} \otimes \dyad{0}{1}),
    \end{align}
    podemos calcular los operadores de Kraus asociados a la evolución parcial del Átomo
    \[ E_\alpha = \bra{\alpha}_C U_{AC} \ket{0}_C = \dyad{0}{0}_A \ip{\alpha}{0}_C + \cos\theta \dyad{1}{1}_A \ip{\alpha}{0}_C + \sin\theta \dyad{0}{1}_A \ip{\alpha}{1}_C \]
    \[
        \implies
        \begin{dcases}
            E_0 = \dyad{0}{0} + \cos\theta \dyad{1}{1} = \begin{pmatrix} 1 & 0 \\ 0 & \cos\theta \end{pmatrix}, \\
            E_1 = \sin\theta \dyad{0}{1} = \begin{pmatrix} 0 & \sin\theta \\ 0 & 0 \end{pmatrix}.
        \end{dcases}
    \]
    Luego, recuperamos la expresión \eqref{eq:I_4_1} mediante
    \begin{align}
        \rho_A' &= E_0 \rho_A E_0^\dagger + E_1 \rho_A E_1^\dagger \\
            &= \begin{pmatrix} 1 & 0 \\ 0 & \cos\theta \end{pmatrix} 
            \begin{pmatrix}
                \rho^A_{00} & \rho^A_{01} \\
                \rho^A_{10} & \rho^A_{11} \\
            \end{pmatrix}
            \begin{pmatrix} 1 & 0 \\ 0 & \cos\theta \end{pmatrix}
            +
            \begin{pmatrix} 0 & \sin\theta \\ 0 & 0 \end{pmatrix}
            \begin{pmatrix}
                \rho^A_{00} & \rho^A_{01} \\
                \rho^A_{10} & \rho^A_{11} \\
            \end{pmatrix}
            \begin{pmatrix} 0 & 0 \\ \sin\theta & 0 \end{pmatrix} \\
            &= \begin{pmatrix} 1 & 0 \\ 0 & \cos\theta \end{pmatrix} 
            \begin{pmatrix}
                \rho^A_{00} & \rho^A_{01} \cos\theta \\
                \rho^A_{10} & \rho^A_{11} \cos\theta \\
            \end{pmatrix}
            +
            \begin{pmatrix} 0 & \sin\theta \\ 0 & 0 \end{pmatrix}
            \begin{pmatrix}
                \rho^A_{01} \sin\theta & 0 \\
                \rho^A_{11} \sin\theta & 0 \\
            \end{pmatrix} \\
            &= 
            \begin{pmatrix}
                \rho^A_{00} & \rho^A_{01} \cos\theta \\
                \rho^A_{10} \cos\theta & \rho^A_{11} \cos^2\theta \\
            \end{pmatrix}
            +
            \begin{pmatrix}
                \rho^A_{11} \sin^2\theta & 0 \\
                0 & 0 \\
            \end{pmatrix} \\
            &= 
            \begin{pmatrix}
                \rho^A_{00} + \rho^A_{11} \sin^2\theta & \rho^A_{01} \cos\theta \\
                \rho^A_{10} \cos\theta & \rho^A_{11} \cos^2\theta \\
            \end{pmatrix}.
    \end{align}
    
    Parar $\rho_A = p_0 \dyad{0}{0} + p_1 \dyad{1}{1} = p_0 \dyad{0}{0} + (1 - p_0) \dyad{1}{1}$, el estado evolucionado será
    \[
        \rho_A' =
        \begin{pmatrix}
            p_0 + p_1 \sin^2\theta & 0 \\
            0 & p_1 \cos^2\theta \\
        \end{pmatrix}
        =
        \begin{pmatrix}
            p_0 + (1 - p_0) \sin^2\theta & 0 \\
            0 & (1 - p_0) \cos^2\theta \\
        \end{pmatrix}.
    \]
    
    El Hamiltoniano asociado a este proceso será
    \begin{align}
        H_{AC} &= \varepsilon (\dyad{1_A 0_C}{1_A 0_C} + \dyad{0_A 1_C}{0_A 1_C} ) + \frac{g}{2} (\dyad{0_A 1_C}{1_A 0_C} + \dyad{1_A 0_C}{0_A 1_C}) \\
            &=
            \begin{pmatrix}
                0 & 0 & 0 & 0 \\
                0 & \varepsilon & \frac{g}{2} & 0 \\
                0 & \frac{g}{2} & \varepsilon & 0 \\
                0 & 0 & 0 & 0 \\
            \end{pmatrix}
    \end{align}
    Encontramos la relación entre constantes exponenciando $H$, observando que
    \[
        U = e^{-iH t} =
        \begin{pmatrix}
            1 & 0 & 0 & 0 \\
            0 & e^{-i \epsilon t} \cos(\frac{gt}{2}) & -ie^{-i \epsilon t} \sin(\frac{gt}{2}) & 0 \\
            0 & -ie^{-i \epsilon t} \sin(\frac{gt}{2}) & e^{-i \epsilon t} \cos(\frac{gt}{2}) & 0 \\
            0 & 0 & 0 & 0 \\
        \end{pmatrix}.
    \]
    Luego, las dos condiciones serán
    \[ \frac{gt}{2} = \theta, \quad t\epsilon = 2k\pi, \quad k \in \mathds{C}. \]
    
\end{enumerate}

\section{Algoritmo de Búsqueda de Grover}
\begin{enumerate}
    
    %-------------------------------------------------------------------------------------------------------
    %   Problema II.1
    %-------------------------------------------------------------------------------------------------------
    \item Consideremos el estado buscado $\ket{B} = \inv{\sqrt{M}} \sum_{f(j) = 1} \ket{j}$ -donde $M$ denota el número de estados $\ket{j}$ tal que $f(j) = 1$-, y el estado ortogonal $\ket{A} = \inv{\sqrt{N - M}} \sum_{f(j) = 0} \ket{j}$. Supongamos que el sistema de $n$ qubits, con $N = 2^n$, se encuentra inicialmente en el estado
    \[ \ket{\Phi} = H^{\otimes n} \ket{0} = \inv{\sqrt{N}} \sum_{j} \ket{j} = \sqrt{\frac{M}{N}} \ket{B} + \sqrt{\frac{N - M}{N}} \ket{A}. \]
    
    Sea $O$ un oráculo, definido por $O \ket{j} = (-1)^{f(j)} \ket{j}$, y el operador de Grover $G = (2 \dyad{\Phi}{\Phi} - \mathds{1}) O$. Escribiendo el estado inicial del sistema como
    \[ \ket{\Phi} = \sin\theta \ket{B} + \cos\theta \ket{A}, \]
    con $\sin\theta = \sqrt{M/N}$, entonces podemos identificar el accionar del oráculo $O$ como una reflexión en el eje $\ket{A}$, ya que
    \[ O \ket{\Phi} = \sin\theta O \ket{B} + \cos\theta O \ket{A} = -\sin\theta O \ket{B} + \cos\theta O \ket{A} = \sin(-\theta) \ket{B} + \cos(-\theta) \ket{A}. \]
    Más aún, si $\ket{\Phi_\perp}$ denota el subespacio ortogonal a $\ket{\Phi}$, entonces
    \[ (2\dyad{\Phi}{\Phi} - \mathds{1}) \ket{\Phi} = \ket{\Phi}, \quad (2\dyad{\Phi}{\Phi} - \mathds{1}) \ket{\Phi_\perp} = -\ket{\Phi_\perp}, \]
    constituyendo una reflexión respecto a $\ket{\Phi}$. Por lo tanto, el operador de Grover sobre $\ket{\Phi}$ representará una rotación en un ángulo $2\theta$, como podemos apreciar en el diagrama representado en la fig. \ref{fig:II_1_1}.
    
    Luego de $k$ iteraciones de la aplicación de $G$, el estado resultante será
    \[ G^k \ket{\Phi} = \sin(\theta + k \ 2\theta) \ket{B} + \cos(\theta + k \ 2\theta) \ket{A}. \]
    Para alcanzar el estado buscado, $G^k\ket{\Phi} = \ket{B}$, por lo que
    \[ \theta + 2k\theta = \theta (1 + 2k) = \frac{\pi}{2} \implies k = \frac{\pi}{4\theta} - \inv{2}. \]
    Bajo la suposición $\theta \ll 1$, $\theta \approx \sin\theta = \sqrt{\frac{M}{N}}$. Por lo tanto,
    \[ k \approx \sqrt{\frac{N}{M}} \frac{\pi}{4} - \inv{2} \sim \order{\sqrt{N}}. \]
    
    \begin{figure}[t]
        \centering
        \begin{tikzpicture}
            \draw[->, gray, thick] (-0.5, 0) -- (5, 0) node[right] {$\ket{A}$};
            \draw[->, gray, thick] (0, -0.5) -- (0, 5) node[above] {$\ket{B}$};
            
            \draw[->, black] (0, 0) -- ({4 * cos(15)}, {4 * sin(15)}) node[right] {$\ket{\Phi}$};
            \draw[->, black] (0, 0) -- ({4 * cos(15)}, {4 * sin(-15)}) node[right] {$O\ket{\Phi} = -\ket{\Phi}$};
            \draw[->, black, dashed] (0, 0) -- ({4 * cos(-75)}, {4 * sin(-75)}) node[right] {$\ket{\Phi_\perp}$};
            \draw[->, black] (0, 0) -- ({4 * cos(45)}, {4 * sin(45)}) node[right] {$G\ket{\Phi}$};
            
            \draw [black, thick] (2, 0) arc [start angle=0, end angle=15, radius=2cm] node [midway, right] {$\theta$};
            \draw [black, thick] (2, 0) arc [start angle=0, end angle=-15, radius=2cm] node [midway, right] {$\theta$};
            \draw [->, black, double, thick] ([shift=(15:2cm)]0, 0) arc [start angle=15, end angle=45, radius=2cm] node [midway, right] {$2\theta$};
        \end{tikzpicture}
        \caption{Representación gráfica del accionar de una iteración del operador de Grover.}
        \label{fig:II_1_1}
    \end{figure}
    
    
    
    %-------------------------------------------------------------------------------------------------------
    %   Problema II.2
    %-------------------------------------------------------------------------------------------------------
    \item Si $H = \hbar \omega (\dyad{B}{B} + \dyad{\Phi}{\Phi})$, es posible generar un operador de evolución unitaria del sistema tal que $U \ket{\Phi} = \exp{-iH t / \hbar} \ket{\Phi} = \ket{B}$, es decir, tal que obtengamos el estado buscado $\ket{B}$ en un solo paso. En la base de los estados $\{ \ket{B}, \ket{A} \}$,
    \[ \ket{\Phi} = \alpha \ket{B} + \beta \ket{A} \implies \dyad{\Phi}{\Phi} = \begin{pmatrix} \alpha^2 & \alpha\beta \\ \alpha\beta & \beta^2 \end{pmatrix}, \]
    por lo que entonces el Hamiltoniano enunciado podrá ser representado matricialmente como
    \begin{align}
        H = \hbar \omega (\dyad{B}{B} + \dyad{\Phi}{\Phi}) &= \hbar\omega \left[ \begin{pmatrix} \alpha^2 & \alpha\beta \\ \alpha\beta & \beta^2 \end{pmatrix} + \begin{pmatrix} 1 & 0 \\ 0 & 0 \end{pmatrix} \right] \\
            &= \hbar\omega \begin{pmatrix} 1 + \alpha^2 & \alpha\beta \\ \alpha\beta & \beta^2 \end{pmatrix} \\
            &= \hbar\omega \begin{pmatrix} 1 + \alpha^2 & \alpha\beta \\ \alpha\beta & 1 - \alpha^2 \end{pmatrix} \\
            &= \hbar\omega \left[ \mathds{1} + \begin{pmatrix} \alpha^2 & \alpha\beta \\ \alpha\beta & -\alpha^2 \end{pmatrix} \right] \\
            &= \hbar\omega \left[ \mathds{1} + \alpha^2 \sigma_z + \alpha\beta \sigma_x \right].
            &= \hbar\omega \mathds{1} + \hbar\omega\alpha \left[ \beta \sigma_x + \alpha \sigma_z \right].
    \end{align}
    
    Luego, utilizando los resultados obtenidos para un operador de rotación en la práctica 3,
    \begin{align}
        U = e^{\frac{-i H t}{\hbar}} = e^{-i\omega t} \left[ \cos(\omega\alpha t) \mathds{1} - i \sin(\omega\alpha t) (\beta \sigma_x + \alpha \sigma_z) \right].
    \end{align}
    Ahora, estudiando la evolución temporal particular del estado inicial $\ket{\Phi}$, e ignorando la fase global $e^{-i\omega t}$,
    \[ U \ket{\Phi} \sim \cos(\omega\alpha t) \ket{\Phi} - i \sin(\omega\alpha t) (\beta \sigma_x + \alpha \sigma_z) \ket{\Phi}. \]
    Evaluando el producto final de matrices,
    \[
        (\beta \sigma_x + \alpha \sigma_z) \ket{\Phi} \equiv
        \begin{pmatrix}
            \alpha & \beta \\
            \beta & -\alpha \\
        \end{pmatrix}
        \begin{pmatrix} \alpha \\ \beta \end{pmatrix}
        =
        \begin{pmatrix} \alpha^2 + \beta^2 \\ \beta\alpha - \alpha\beta \end{pmatrix}
        =
        \begin{pmatrix} 1 \\ 0 \end{pmatrix}
        \equiv
        \ket{B},
    \]
    entonces
    \[ U \ket{\Phi} \sim \cos(\omega\alpha t) \ket{\Phi} - i \sin(\omega\alpha t) \ket{B}. \]
    Por lo tanto, para que resulte en el estado deseado $\ket{B}$,
    \[ \omega\alpha t = \frac{\pi}{2} \implies t = \frac{\pi}{2\omega\alpha}. \]
    
    El problema de búsqueda se reduce entonces al problema de simular el Hamiltoniano $H$, siguiendo el algoritmo de Trotter:
    \[ U = e^{\frac{-iHt}{\hbar}} = \qty(e^{\frac{-iHt}{\hbar k}})^k \approx \qty(e^{-i\omega\alpha \frac{t}{k} \dyad{\Phi}{\Phi}} \ e^{-i\omega\alpha \frac{t}{k} \dyad{B}{B}})^k, \quad k \gg 1. \]
    \[ \implies k \sim \order{N}. \]
    
\end{enumerate}

%\nocite{*}
%\printbibliography
\end{document}